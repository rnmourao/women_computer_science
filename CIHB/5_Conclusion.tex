\section{Conclusion}\label{sec:conclusion}%

In recent years, the area of Computer Science has had the participation of few women in the profession, showing that girls have not shown interest in majoring or becoming professionals in this field. Given this scenario, research aimed at understanding their motives for not choosing a major or career in computers is important to promoting actions that will reduce the disparity between boys and girls entering the field, by increasing participation of the girls.

Data mining was performed in the answers from a questionnaire applied from 2011 to 2014, and the Apriori algorithm showed that the single most important factor for a Middle or High School girl deciding whether to pursue a degree in a CS major is her family's approval of this choice. This result provides insights to guide further actions to reduce the gender gap in the field.

Future works include: improving the questionnaire; applying the analysis to data for  Middle and High School boys, for comparison, as well as to students who have enrolled in a CS major; and investigating other data mining approaches for knowledge discovery.

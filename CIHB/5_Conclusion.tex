\section{Conclusion}\label{sec:conclusion}%

In recent years, the gender gap in the field of Computer Science has widened and girls have not shown interest in majoring or becoming professionals. In order to address the issue, we performed a research to better understand this scenario and its causes, investigating the girls' perspectives on a Computer Science major. The knowledge discovered could to guide and support actions that will reduce the disparity by increasing participation of the girls.

Data mining and statistical analysis were applied to the data from a questionnaire done from 2011 to 2014, and results showed that the single most important factor for a Middle or High School girl deciding whether to pursue a degree in a CS major is her family's approval of this choice. Insights on the importance of other factors, such as career opportunities and Math requirements for a major were also obtained. The analysis and the mining process sparked new questions and discussions.

After this research, the project \texttt{Meninas.comp} changed some of its approaches. We created activities targeted exclusively at Middle School girls; provided  lectures specifically about jobs in computing and lectures featuring important women in the field; proposed activities involving Mathematics, Computing and games. We will be watching the results on these actions to see if they will help reduce the gender gap.

Future works include: improving the questionnaire; applying the analysis to data for Middle and High School boys, for comparison, as well as to students who have enrolled in a CS major; and investigating other data mining approaches for knowledge discovery.

\section{Data on Brazilian Middle and High School Girls}\label{sec:mining}%

In this paper we used two different analysis: first, statistical analysis where the collection data was grouped according to computing science interest; and second, the association rules mining for discovering the relation among the questions and the girls which response computing science interest yes.

\subsection{Statistical Analysis}\label{sec:mining:stat}%
The data for all years was consolidated in a single spreadsheet, which was then processed in the \texttt{R} programming language. The questionnaires, data and scripts used on this work are freely available online on github Project \footnote{\url{http://goo.gl/oJYrjh}}.

The script cleans up the data (empty columns, whitespaces, etc.) and begins to process it for analysis. Then we selected the data subset for group of interest: Middle or High School girls who have answered if they are interested in a CS major.

Figure~\ref{fig:FieldOfInterest} shows the respondents' interest in different fields of study. These results indicate that, throughout the years, the percentages for each choice remains more or less the same with an average of $41\%$ for \emph{Biology-Health Sciences}, $22\%$ for \emph{Exact Sciences} and $33\%$ for \emph{Human Sciences}. As can be seen, the field of Physical Sciences is the less frequently chosen area in every year analyzed.

\begin{figure}[h!]%
\includegraphics[width=\textwidth]{img/{FieldOfInterest}.pdf}%
\caption{Respondents' fields of interest.}%
\label{fig:FieldOfInterest}%
\end{figure}%

Figure~\ref{fig:WouldEnrollInCS} shows the respondents' interest in enrolling in a Computer Science major. On average, the data shows that $31\%$ of the girls \emph{have interest} while $28\%$ \emph{have no interest} and $41\%$ \emph{have doubt}. However, it is important to highlight that the year 2011 was different from the previous ones, since information was collected with students from the fifth grade of elementary school to the third year of high school. During the years 2012, 2013 and 2014, due to a reduction of members on the Project Meninas.comp' team, we restricted our targeted participants to students of the ninth year of middle school and the high school.

\begin{figure}[h!]%
\includegraphics[width=\textwidth]{img/{WouldEnrollInCS}.pdf}%
\caption{Respondents interested in Computer Science.}%
\label{fig:WouldEnrollInCS}%
\end{figure}%

To understand the profile of students who are interested in Computer Sciences, this section presents the results of responses to various questions from the research questionnaire, in relation to the variable, interest in computers (\textit{Would you major in Computer Science?}).

Figure~\ref{fig:EducationalStage}  presents the results of answers to the question whether or not the student is interested in doing a CS major. Results are organized by grade level in school. Two sets of findings are highlighted: i) Third year high school students had the lowest rate of positive responses; ii) Elementary school students had the highest rate of positive responses.


\begin{figure}[h!]%
\centering%
\includegraphics[width=.6\textwidth]{img/{plot.Educational.Stage}.pdf}%
\caption{Results to the question \textit{Would you major in Computer Sciences?}, by Grade level.}%
\label{fig:EducationalStage}%
\end{figure}%

Figure~\ref{fig:MaleMajority}, on the other hand, shows the synthesis of answers to the question \textit{Are the majority of CS students male?}. As can be observed, the majority of girls perceived that more boys than girls major in CS. In the girls group interested in CS  (Would.Enroll.In.CS = yes), the answer to the question if there are more boys in CS had the highest rate of response \textit{No}(CS.Has.A.Majority.Male=no).

\begin{figure}[h!]%
	\centering
    \begin{subfigure}[t]{0.48\textwidth}
		\includegraphics[width=\textwidth]{img/{plot.CS.Has.A.Male.Majority}.pdf}%
		\caption{Are the majority of students in CS male?}%
		\label{fig:MaleMajority}%
	\end{subfigure}
	~
    \begin{subfigure}[t]{0.48\textwidth}
		\includegraphics[width=\textwidth]{img/{plot.Family.Approves.CS.Major}.pdf}%
		\caption{Does your family approve your interest in a CS major?}%
		\label{fig:plot.Family.Approves.CS.Major}%
	\end{subfigure}

	\begin{subfigure}[t]{0.48\textwidth}
		\includegraphics[width=\textwidth]{img/{plot.CS.Has.Low.Employability}.pdf}%
		\caption{Is it difficult to find employment in CS after graduating?}%
		\label{fig:plot.CS.Has.Low.Employability}%
	\end{subfigure}
	~
	\begin{subfigure}[t]{0.48\textwidth}
		\includegraphics[width=\textwidth]{img/{plot.CS.Work.Has.Long.Hours}.pdf}%
		\caption{Do people who work in CS have long work hours?}%
		\label{fig:plot.Low.Leisure}%
	\end{subfigure}

	\begin{subfigure}[t]{0.48\textwidth}%
		\includegraphics[width=\textwidth]{img/{plot.CS.Is.Prestigious}.pdf}%
		\caption{Does working in CS bring prestige?}%
		\label{fig:plot.CS.Is.Prestigious}%
	\end{subfigure}%
	~
	\begin{subfigure}[t]{0.48\textwidth}%
		\includegraphics[width=\textwidth]{img/{plot.CS.Provides.Good.Wages}.pdf}%
		\caption{Do people who work in CS earn well?}%
		\label{fig:plot.CS.Provides.Good.Wages}%
	\end{subfigure}%
	\caption{Relations between \emph{Would.Enroll.In.CS} attributes and other variables per poll question.}%
	\label{fig:CS-relations}%
\end{figure}%

Results to the next question, \textit{Do people who work in CS have long work hours?}, can be seen in Figure ~\ref{fig:plot.CS.Work.Has.Long.Hours}. The majority of girls responded negatively to the question. Interestingly, the group of girls who reported being interested in CS, responded positively to this question at a higher rate than the other groups. The group of girls who reported not being interested in CS had a higher percentage of answers in the Maybe category.

In relation to the question \textit{Would your family like it if you took the college entrance exam for a major in CS?}, Figure~\ref{fig:plot.Family.Approves.CS.Major} shows the results. The importance of family approval is clear. Among the girls who answered that they would be interested in doing a major in CS (Would.Enroll.In.CS = yes), the majority reported having their family's approval (Family.Would like.CS = yes). And the girls who reported not being interested in CS  (Would.Enroll.In.CS = No) had the highest rate of  negative responses to family's approval (Family.Would like.CS = no).

The next question analyzed was: \textit{Is it difficult to find employment in computers after graduating?}. Figure ~\ref{fig:plot.CS.Has.Low.Employability} presents the results to this question. As can be observed, respondents reported that they did not think it was difficult to find employment in the area of computers as the response, Difficult.Employment = No, had the highest rate of response in all of the groups analyzed.

Results to the next question, \textit{Do people who work in CS have very little leisure time?}, can be seen in Figure ~\ref{fig:plot.CS.Work.Has.Long.Hours}. The majority of girls responded negatively to the question (Little.Leisure = no). Interestingly, the group of girls who reported being interested in CS, responded positively to this question at a higher rate than the other groups. The group of girls who reported not being interested in CS had a higher percentage of answers in the Maybe category (Little.Leisure = maybe).

Response results to the following question, \textit{Does working with computers bring prestige?},  are shown in Figure ~\ref{fig:plot.CS.Is.Prestigious}. The majority of girls reported positively to this question (Brings.Prestige = yes). The group with the lowest yes response rate to this question was the group of girls who were not interested in CS, although, percentage wise, this group had a high rate of maybe answers (Brings.Prestige = Maybe).

In relation to the question, \textit{Do people who work in CS earn well?}, Figure~\ref{fig:plot.CS.Provides.Good.Wages} presents the response results. The majority of girls responded yes (Earn.Well= yes). However, it is worthy to note that there was also a high rate of responses in the category of maybe (Earn.Well = maybe) in all of the groups of girls surveyed. Specifically, for the group of girls who were not interested in CS, the maybe response was the one most frequently reported.

In relation to the question, \textit{Does a higher education major in CS require little math?}, responses are shown in Figure~\ref{fig:plot.CS.Uses.Little.Math}. As can be observed, all of the students know that a major in CS requires much math, as the response, Little.Math = no, was the most frequent response in all of the groups surveyed.

\begin{figure}%
\centering%
\includegraphics[width=.6\textwidth]{img/{plot.CS.Uses.Little.Math}.pdf}%
\caption{Results to the question \textit{Does a higher education major in CS require little math?}}%
\label{fig:plot.CS.Uses.Little.Math}%
\end{figure}%

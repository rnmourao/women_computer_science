\documentclass[preprint,12pt]{elsarticle}%

%% Use the option review to obtain double line spacing
%% \documentclass[authoryear,preprint,review,12pt]{elsarticle}

%% Use the options 1p,twocolumn; 3p; 3p,twocolumn; 5p; or 5p,twocolumn
%% for a journal layout:
%% \documentclass[final,1p,times]{elsarticle}
%% \documentclass[final,1p,times,twocolumn]{elsarticle}
%% \documentclass[final,3p,times]{elsarticle}
%% \documentclass[final,3p,times,twocolumn]{elsarticle}
%% \documentclass[final,5p,times]{elsarticle}
%% \documentclass[final,5p,times,twocolumn]{elsarticle}

%% \usepackage{amssymb}
%% \usepackage{amsthm}

\journal{Computers In Human Behavior}%

%%%%%%%%%%%%%%%%%%%%%%%%%%%%%%%%%%%%%%%%%%%%%%
\usepackage[utf8]{inputenc}%                 %
\usepackage[T1]{fontenc}%                    %
\usepackage[hidelinks]{hyperref}%            %
\usepackage{subcaption}%                     %
\graphicspath{{../github/img/}}%       %
%                                            %
\usepackage{xcolor}%                         %
\newcommand{\gnramos}[1]{\textcolor{red}{#1}}%
%%%%%%%%%%%%%%%%%%%%%%%%%%%%%%%%%%%%%%%%%%%%%%

\begin{document}

\begin{frontmatter}

%% Title, authors and addresses

%% use the tnoteref command within \title for footnotes;
%% use the tnotetext command for theassociated footnote;
%% use the fnref command within \author or \address for footnotes;
%% use the fntext command for theassociated footnote;
%% use the corref command within \author for corresponding author footnotes;
%% use the cortext command for theassociated footnote;
%% use the ead command for the email address,
%% and the form \ead[url] for the home page:
%% \title{Title\tnoteref{label1}}
%% \tnotetext[label1]{}
%% \author{Name\corref{cor1}\fnref{label2}}
%% \ead{email address}
%% \ead[url]{home page}
%% \fntext[label2]{}
%% \cortext[cor1]{}
%% \address{Address\fnref{label3}}
%% \fntext[label3]{}

\title{Brazilian School Girls' Perspectives on a Computer Science Major: Mining Association Rules}

\author{Maristela Terto de Holanda\corref{cor1}\fnref{label1}}%
\ead{mholanda@unb.br}%
\ead[url]{http://cic.unb.br}%
\author{Roberto Mourão}%
\author{Aleteia Araújo}%
\author{Maria Emília Walter}%
\author{Guilherme N. Ramos}%


%% use optional labels to link authors explicitly to addresses:
%% \author[label1,label2]{}
%% \address[label1]{}
%% \address[label2]{}

\address{Department of Computer Science, University of Brasília, Brasília, Brazil}%

\begin{abstract}
The field of Computer Science (CS) has been of little interest for girls straight out of high school, when considering undergraduate majors in Brazil. At the University of Brasília’s CS Department, female students compose less than 10\% of the student body. In an effort to understand the girls’ lack of interest in computer related courses, we applied an anonymous questionnaire, from 2011 to 2014, regarding their perceptions of the field.  The participants were 3707 females students who completed an anonymous questionnaire. We applied Association Rules in Data Mining to analyze the responses discovering that the family's approval of such choice is one of the most important factors. The knowledge gained through this study could guide future research on the matter and guidelines for motivating girls to pursue careers in Computer Science.
\end{abstract}

\begin{keyword}
%% keywords here, in the form: keyword \sep keyword
computer science \sep gender \sep girls \sep women \sep female student \sep data mining \sep association rules \sep apriori%

%% PACS codes here, in the form: \PACS code \sep code

%% MSC codes here, in the form: \MSC code \sep code
%% or \MSC[2008] code \sep code (2000 is the default)
\end{keyword}

\end{frontmatter}

%% \linenumbers

%% main text
\section{Introduction}\label{sec:intro}%

%Begin Maristela
In Brazil, the choice of an undergraduate major in the area of Computer Sciences is not among the top choices for girls in high school when contemplating a career. As \cite{maia_2016} presents, between 2000 and 2013 in Brazil, an average of only 17\% of all graduates in various Computer Science majors were women. This research covered majors in Computer Science, Computer Engineering, and Information Systems, among others. Particularly, in the Federal District, at the University of Brasilia, which currently has approximately 30,000 students enrolled in undergraduate programs, the reality is even worse, where in the past 10 years, according to \cite{couto_2014} only 10\% were women.

	Responding to the low incidence of women in Computer Science majors, recently researchers have given much thought about how to improve this scenario and proposed strategies to encourage girls to pursue a profession in the Computer Sciences   \cite{cohoon_2002} \cite{couto_2014}  \cite{gurer_2002}  \cite{maia_2016}. Brazil and other countries have developed initiatives to debate this issue. Specifically, the Institute of Electrical and Electronics Engineers (IEEE) has a program which address the problem: the IEEE Women in Engineering (WIE) \cite{wie2017}. The WIE is a major professional and international organization dedicated to promoting women scientists and engineers. Another prominent program in promoting women in the area of Computers is ?Girls who Code? \cite{girlsWC_2017}, which has over 40,000 members and various initiatives to increase the participation of girls in Computer Sciences over various regions in the United States. Another initiative from the United States is the ?Grace Hopper Celebration of Women in Computer Sciences? event, which is the biggest event worldwide for discussing the theme of women in the field. In 2016 alone 15,000 people from 87 countries participated in the 700 presentations \cite{GHC_2017}.

	In Brazil, since 2007, the Brazilian Society of Computing Conference held the Women in Information and Technology Workshop (WIT), to discuss the theme. Brazilian governmental agencies, such as the Ministry of Science and Technology released calls for submissions of research projects specifically related to the education of girls in Computing or Physical Science majors \cite{cnpq_2017}. Aiming to gather information about the perceptions of high school girls regarding computer science, the Department of Computer Science at the University of Brasilia, developed the project, Meninas.comp: computação  tambem e coisa de menina, Girls.comp: computer science is a girl thing too.

%End Maristela

\gnramos{Falar da análise realizada (Roberto e Guilherme)}%

Between the years of 2012 and 2014, we contacted thousands of girls in Middle and High School to investigate the relationship between their intention to apply for an undergraduate course in Computer Science and their affinity with the field and computing tools. We used the Apriori algorithm on the collected data, searching for interesting association rules and insights on the girls' interest in CS and their background.

This following sections of this work are organized as follows: Section~\ref{sec:background} presents related work and background information on this approach, Section~\ref{sec:mining} describes details of the data mining applied, Section~\ref{sec:results} provides our experimental results and our finds and Section~\ref{sec:conclusion} presents concluding remarks.
%
\section{Background \& Related Works}\label{sec:background}%

\subsection{Girls in Computer Science}\label{subsec:background:girls}%
2. Meninas e a Computação (Maristela)

Middle School: Ensino Fundamental II
High School: Ensino Médio

\subsection{Data Analysis}\label{subsec:background:data}%

Data analysis includes, among other things, procedures for analyzing data and techniques for interpreting their results~\cite{Tukey1962}.

\gnramos{Breve relação das técnicas utilizadas}%
ANOVA ~\cite{Hastie2009}


Frequently, data may be usefully and accurately treated by the analysis of variance (through separation of the variance attributes to a group of causes from that attributed to other groups)~\cite{Fisher1934}. This \emph{analysis of variance} (ANOVA)can be achieved through several approaches, for example through a hypothesis test comparing the two models, considering the null hypothesis that both models fit the data equally well in contrast to  the full model is superior~\cite{James2013}.

\gnramos{Trabalhos correlacionados}%


%
\section{Data Collection and Analysis}\label{sec:data}
The enrollment of female students in Computer Science majors is decreasing every year and one of the biggest challenges in addressing this is to discover what motivates girls to avoid a CS major. Some issues have been outlined in the literature, such the impact of associated stereotypes; but our research intends to further investigate the women's perceptions of the Computer Science field by looking at the prospective enrollees: girls in high school. There is a significant lack of research on this subject in Brazil, and we believe such analysis could aid in the proposal of policies for increasing female participation in the field.

\subsection{Survey}
The members of the \texttt{Meninas.comp} from the University of Brasilia developed a questionnaire to inquire about female high school students' perceptions of the Computer Science field. It included personal profile questions, such as sex, school year, field of interest for a college education, interest in a career in CS, and others; as well as more general questions related to computers, such as: where the student uses computers and for what kind of tasks.

From 2011 to 2014, the questionnaires were given to female high school students in the Federal District, during during Brazil's Ministry of Science and Technology's National Science and Technology Week\footnote{\url{http://semanact.mcti.gov.br/}} activities.

% Finally, the questionnaire comprised specific (yes, no, maybe) questions regarding the prestige of working with computers and wages earned working with computers, among others:
% Does a university computer science course only teach how to use software?;
% Does a university computer science course require few math skills?;
% Are the majority of Computer Science majors male?;
% Is it necessary to know how to use a computer to enroll in a university computer course?;
% Is it necessary to work in computer sciences to enroll in a computer course?;
% Would your family like you to take the college entrance exam for Computer Sciences?;
% Is it difficult to find work in computer sciences after graduating?;
% Do people who work in computer sciences have little leisure time?;
% Does working in computer sciences allow you to exercise your creativity?;
% Is working in computer sciences prestigious?; and
% Does working in computer sciences pay well?.


% \subsubsection{Research Data}%

There were 1,821 responses in 2011, 944 in 2012, 517 in 2013, and 425 in 2014; adding up to a total of 3,707 completed questionnaires. The decrease of respondents in the period is due to the project's context, the amount of work possible is directly affected by the number of volunteers working on it. In 2011, there were 5 Professors and 10 students members, but they dwindles to only 1 Professor and 5 students (working few hours) in 2014. The collected data was consolidated in a spreadsheet and analyzed.

% Subsequently, in the data preparation phase, we received the collection data in 4 spreadsheets, one by year. Then, the columns of the spreadsheets were reorganized. This reorganization was applied to the 4 spreadsheets, corresponding to 2011, 2012, 2013, and 2014. The organization of the answers were tabulated and served as the source for the creation of a single table, comprising 35 attributes. The main attributes used in the analysis of this paper were: Bring\_Prestige, Computer\_Friends, Computer\_Lan\_House, Computer\_Relatives , CS\_Interest, Family\_Approval, Good\_Salary, Low\_Leisure, Use\_Creativity, Computer\_Home, Computer\_Library, Computer\_School,  CS\_Choice, Educational\_Stage, Field\_Interest, Has\_Low\_Math, Low\_Employability, Man\_Majority and Year.

% Finally, the data were filtered so that of the 3,707 questionnaires,  researchers excluded those answered by students who: were already in higher education;  were male;  failed to answer the key question: \textit{Are you interested in doing a major in Computer Sciences?}. This left 3,161 questionnaires to be analyzed.


\subsection{Data Analysis}\label{sec:analysis:related}%

Data analysis includes, among other things, procedures for analyzing data and techniques for interpreting their results~\cite{Tukey1962}, while Data Mining is the process of discovering insightful patterns and predictive models from data~\cite{Zaki2014}, in an effort to make sense of usually large amounts of information in some domain~\cite{Cios2007}. Our primary focus is the gender gap in pSTEM careers, so our study aims to characterize the profiles of girls who intend to enroll in undergraduate studies, especially those interested in Computer Science.

One of the possible approaches to finding interesting relationships in data is \emph{association rules mining}~\cite{Cios2007}, which produces easily understandable results as rules states as ``\emph{if A occurred, then B occurs}''. For example, it is likely that a rule ``\emph{if a girls sees the CS field as boring, she will not enroll in a CS major}'' is found. This processing may require a lot of resources, and a computationally feasible solution is the \emph{Apriori} algorithm, which uses only the itemsets found large in the previous pass to generate candidate itemsets~\cite{Agrawal1994}, and produces the rules with the highest \emph{confidence} (how often the rule has been found to be true), despite their \emph{support} (number of occurrences)~\cite{taniar_exception_2008}. The confidence is the conditional probability $P(B|A)$, i.e., the probability of \emph{A} will occur, since \emph{B} occurred~\cite{Hastie2009}. In order to select  which rules are more interesting rules, we consider their \emph{lift}, which represents the level of association between the antecedent and the consequent~\cite{tan2006introduction}.

Thus, we search for insights on students' motivation for academic studies in Computer Science, with a clear gender bias (looking at only females) and analyzing a large data set of Brazilian students.
%
\newcommand{\subfig}[1]{%
    \begin{subfigure}[t]{0.48\textwidth}%
		\includegraphics[width=\textwidth]{{Would.Enroll.In.CSx#1}.pdf}%
		\caption{}%
		\label{fig:#1}%
	\end{subfigure}%
}%
\newcommand{\fig}[5]{%
	\begin{figure}[h!]%
		\centering%
		\subfig{#2}%
		~
		\subfig{#3}%

		\subfig{#4}%
		~
		\subfig{#5}%
		\caption{Relations between \emph{Would.Enroll.In.CS} attributes and other variables.}%
		\label{fig:#1}%
	\end{figure}%
}%

%%%%%%%%%%%%%%%%%%%%%%%%%%%%%%%%%%%%%%%%%%%%%%%%%%%%%%%%%%%%%%%%%%%%%%%%%%%%%%%%
%%%%%%%%%%%%%%%%%%%%%%%%%%%%%%%%%%%%%%%%%%%%%%%%%%%%%%%%%%%%%%%%%%%%%%%%%%%%%%%%
%%%%%%%%%%%%%%%%%%%%%%%%%%%%%%%%%%%%%%%%%%%%%%%%%%%%%%%%%%%%%%%%%%%%%%%%%%%%%%%%
\section{High School Girls' Perceptions On Computer Science}\label{sec:perception}%

We took two approaches for analysis: statistical analysis for a better understanding of the data collected and association rules mining searching for interesting relationships between the data and the girl's interest in pursuing a CS major.

\subsection{Statistical Analysis}\label{sec:perception:stat}%
The data for all years was consolidated in a single spreadsheet, which was then processed in the \texttt{R} programming language. The questionnaires, data and script used on this work are freely available online\footnote{\url{http://goo.gl/oJYrjh}}.

The preprocessing step cleans up the data (empty columns, whitespaces, etc.) and discards the data not in our subset of interest: Middle or High School girls who have answered whether they are interested in a CS major.

Figure~\ref{fig:Field.Of.Interest} shows the respondents' interests in different scientific fields of undergraduate studies. The data indicates that, throughout the years, the percentages for each choice remains more or less the same, roughly around a value of $41\%$ for \emph{Biology-Health Sciences}, $22\%$ for \emph{Exact Sciences} and $33\%$ for \emph{Human Sciences}. It is clear that the field related to Computer Science (Exact Sciences) is the least interesting for all years.

\begin{figure}[h!]%
\includegraphics[width=\textwidth]{{Field.Of.Interest}.pdf}%
\caption{Respondents' interests in scientific fields.}%
\label{fig:Field.Of.Interest}%
\end{figure}%

Figure~\ref{fig:Would.Enroll.In.CS} shows the respondents' interests in enrolling in a Computer Science major. On average, the data shows that $31\%$ of the girls \emph{have interest} while $28\%$ \emph{have no interest} and $41\%$ \emph{have doubt}. The data for the year 2011 differs a little from the others because the respondents for that year included Middle school students of all ages (from 5th to 9th graders). In the other years, only students from 9th grade or higher were surveyed due to limited human resources.

\begin{figure}[h!]%
\includegraphics[width=\textwidth]{{Would.Enroll.In.CS}.pdf}%
\caption{Respondents interested in Computer Science.}%
\label{fig:Would.Enroll.In.CS}%
\end{figure}%

In order to investigate the profile of the students who are interested in Computer Sciences, we look at how the answers to the question \textit{Would you major in Computer Science?} relate to the other questions. Figure~\ref{fig:Educational.Stage} presents how students in different grades responded. The data shows that 12th graders had the lowest ratio of positive responses and that middle schoolers had the highest. \gnramos{This indicates an interesting research question: \emph{why do girls lose interest in CS as they grow older?}}

\begin{figure}[h!]%
\centering%
\includegraphics[width=.6\textwidth]{{Would.Enroll.In.CSxEducational.Stage}.pdf}%
\caption{Results to the question \textit{Would you major in Computer Sciences?}, by Grade level.}%
\label{fig:Educational.Stage}%
\end{figure}%

Figures~\ref{fig:CS-relations:1} to~\ref{fig:CS-relations:2} present the relationships of several variables observed. The titles indicate the question asked, and the legend on the right side the answers given; the plotted bars indicated how these answers relate do the respondents interest in enrolling in a CS Major.

\fig{CS-relations:1}
    {CS.Only.Teaches.To.Use.Software}
    {CS.Uses.Little.Math}
    {Most.CS.Students.Are.Male}
    {CS.Requires.Knowledge.In.Computers}%

Figure~\ref{fig:CS.Only.Teaches.To.Use.Software} and~\ref{fig:CS.Uses.Little.Math} show that the girls clearly know that CS majors teach more than just using softwares and requires Mathematical knowledge. Figure~\ref{fig:Most.CS.Students.Are.Male} shows that the majority of girls perceived that there are more boys than girls in CS majors. Within the group of girls that are not considering a CS major (\texttt{Would.Enroll.In.CS = no}), it is quite clear that they perceive the field to be dominated by men. \gnramos{This indicates an interesting research question: \emph{are girls not interested in CS because most students are boys?}} Figure~\ref{fig:CS.Requires.Knowledge.In.Computers} indicates that most of them perceive previous knowledge in using computers as a requirement for enrolling in a CS major, \gnramos{presenting another interesting question: \emph{how much knowledge using computers is required to enroll?}}

\fig{CS-relations:2}
    {Higher.Education.Required.To.Work.In.CS}
    {Family.Approves.CS.Major}
    {CS.Has.Low.Employability}
    {CS.Work.Has.Long.Hours}%

Figure~\ref{fig:Higher.Education.Required.To.Work.In.CS} shows that the girls believe that a degree is required for a career in Computer Science. Figure~\ref{fig:Family.Approves.CS.Major} emphasizes the importance of family approval. Among the girls who answered that they would be interested in enrolling a major in CS, the majority reported having their family's approval (\texttt{Family.Approves.CS.Major = yes}); and the girls who reported not being interested had the highest rate of negative responses. Looking at Figures~\ref{fig:CS.Has.Low.Employability} and~\ref{fig:CS.Work.Has.Long.Hours}, and considering the data for girls who did not say they wish to enroll, we wonder \gnramos{\emph{why won't they pursue a career without long hours that they believe is full of opportunities?}}
Interestingly, the group of girls who reported being interested in CS, responded positively to \emph{long hours} at a higher ratio than the other groups.

\fig{CS-relations:3}
    {CS.Fosters.Creativity}
    {CS.Is.Prestigious}
    {CS.Provides.Good.Wages}
    {CS.Enables.Interdisciplinary.Experiences}%

Figures~\ref{fig:CS.Fosters.Creativity} and \ref{fig:CS.Enables.Interdisciplinary.Experiences} clearly show that the girls perceive CS as a creative field with various interdisciplinary possibilities. Figure~\ref{fig:CS.Is.Prestigious} has a favorable perception; the group with the lowest \emph{yes} response ratio to this question was that of girls who were not interested in CS, despite this group having a high ratio of \emph{maybe} replies. Figure~\ref{fig:CS.Provides.Good.Wages} indicates that the majority of girls think there are good salaries in the field, but it is worth to note that there was also a large part of them also responded \emph{maybe}, specially in the group of girls who were not interested in CS.

The remaining questions simply inquire where the girls use computers and what software tools they use. Almost all use a computer at home and most also use is at a relative or friend's house; about half use them at school, and the vast majority does not use a computer at work, at the library, or in digital inclusion centers. Considering tools, most have used text or image editors, but more than half have not used spreadsheets and very few have used databases.

% \fig{CS-relations:4}{Uses.Computer.At.Home}{Uses.Computer.At.Relatives.House}use.{Uses.Computer.At.Friends.House}{Uses.Computer.At.School}%
% \fig{CS-relations:5}{Uses.Computer.At.Work}{Uses.Computer.At.Lan.House}{Uses.Computer.At.Library}{Uses.Computer.At.Digital.Inclusion.Center}%

% \fig{CS-relations:6}{Has.Used.Text.Editor}{Has.Used.Image.Editor}{Has.Used.Spreadsheet}{Has.Used.Database}
% \fig{CS-relations:7}{Has.Used.Internet}{Has.Used.Social.Network}{Has.Used.Games}{Has.Used.Email}
% \fig{CS-relations:8}{Has.Used.For.Creating.Web.Pages}{Has.Used.For.Development}{Has.Used.Other.Softwares}%{Has.Used.Other.Softwares}%


%%%%%%%%%%%%%%%%%%%%%%%%%%%%%%%%%%%%%%%%%%%%%%%%%%%%%%%%%%%%%%%%%%%%%%%%%%%%%%%%
%%%%%%%%%%%%%%%%%%%%%%%%%%%%%%%%%%%%%%%%%%%%%%%%%%%%%%%%%%%%%%%%%%%%%%%%%%%%%%%%
%%%%%%%%%%%%%%%%%%%%%%%%%%%%%%%%%%%%%%%%%%%%%%%%%%%%%%%%%%%%%%%%%%%%%%%%%%%%%%%%
\subsection{Mining Association Rules}\label{sec:results}%
In order to try to understand the profiles of girls intending to enroll in undergraduate courses, especially those interested in Computer Science, we applied Apriori's association rule algorithm to the data, with the minimum confidence level equal to 50\%, and with a maximum number of 3 items in an itemset. A filter was applied to select only the rules involving the variable \emph{Would.Enroll.In.CS}, which indicates the respondent is interested in pursuing a CS degree, on the rules' right-hand sides. The rules were analyzed considering \emph{support}, \emph{confidence} and \emph{lift} metrics. The association rule mining resulted in 32 rules involving the students' interest in Computer Science. Their details are presented in Figure~\ref{fig:apriori.rules}.

\begin{figure}%
\centering
\includegraphics[scale=0.5]{{apriori.rules}.pdf}%
\caption{Rules with highest confidence, ordered by lift.}%
\label{fig:apriori.rules}%
\end{figure}%

We investigated ten main rules, which are with higher lift (stronger red color). Thus we looked into the subset of rules with lift greater than 50\%. This way we have more than 50\% of true in association rule. Figure~\ref{fig:apriori} presents the found rules with higher confidence, ordered by lift. Similarly, statistical analysis, the clearest information from this result was the presence of the \mbox{\emph{Family.Approves.CS.Major}} attribute in all rules.

\begin{figure}%
\includegraphics[width=\textwidth]{apriori-table}%
\caption{Rules with highest confidence, ordered by lift.}%
\label{fig:apriori}%
\end{figure}%

The rule with the highest lift (first in Figure~\ref{fig:apriori}) also had the highest confidence level; so we can say that \gnramos{there is 68\% chance for the rule ``\emph{if the respondent believes that she has the family's approval and higher education is required to work with CS, then she would enroll in a CS major}''.} The value of 1.8 for lift indicates that there is an 80\% chance that the antecedents and the consequent for this rule are correlated, which is a very intriguing discovery. Therefore, girls who are interested in computing, with family approval, know that to work in computing needs a high level of knowledge.

The rule with the second highest lift means that 65\% of times ``\emph{if the respondent believes that she has the family's approval and she has used games, then she would enroll in a CS major}''. Again, there is a very strong correlation between the antecedents and the consequent.{there is 65\% chance for the rule ``\emph{if the respondent believes that she used games, then she would enroll in a CS major}''.} The value of 1.7 for lift indicates that there is a 70\% of chance that the antecedents and the consequent for this rule are correlated.

The third rule with the highest lift states that ``\emph{if the respondent believes that she has the family's approval and she used computer at a relatives' house, then she would enroll in a CS major}''. There is 64\% chance for the antecedent rule then the consequent rule to happen. The value of 1.7 for lift indicates that there is a 70\% of chance that the antecedents and the consequent for this rule are correlated.

\begin{figure}%
\centering
\includegraphics[scale=0.8]{{freq.Family.Approves.CS.Major}.pdf}%
\caption{Respondents by answer for family approval and in interest in CS.}%
\label{fig:freq.Family.Approves.CS.Major}%
\end{figure}%

The next tree rules, include the attributes with CS.Enables.Interdisciplinary.Experiences=Yes, Uses.Computer.At.Work=No and CS.Fosters.Creativity=Yes. These rules have a 63\% of occurring. Again, there is a very strong correlation between the antecedents and the consequent, lift 1.7 (70\%). About the use computer at work, there are a small number of respondents who work (15\%), which can make interpreting difficult. What if the girls who answered that they do not use computers at work do not work? The dubious nature of this question may obscure the understanding of the rule.

The Seventh rule with highest lift states that ``\emph{if the respondent believes that she has the family's approval then she would enroll in a CS major}'', indicating that the family's approval is an important factor. This rule's confidence and lift are only a fraction smaller than the previous ones, while its support is slightly larger. Figure~\ref{fig:freq.Family.Approves.CS.Major} presents the data for this in more detail.

Thee same analysis was applied to the remaining rules in Figure~\ref{fig:apriori}, and this investigation showed that the other attributes, such as \emph{Uses.Computer.At.Library=No}, \emph{Has.Used.Internet=Yes} or \emph{CS.Is.Prestigious=Yes} have low significance when compared to \emph{Family.Approves.CS.Major}. This implies that, for the respondent girls, their family's opinion is the single most important factor when deciding to apply (or not) for a CS course.


%%%%%%%%%%%%%%%%%%%%%%%%%%%%%%%%%%%%%%%%%%%%%%%%%%%%%%%%%%%%%%%%%%%%%%%%%%%%%%%%
%%%%%%%%%%%%%%%%%%%%%%%%%%%%%%%%%%%%%%%%%%%%%%%%%%%%%%%%%%%%%%%%%%%%%%%%%%%%%%%%
%%%%%%%%%%%%%%%%%%%%%%%%%%%%%%%%%%%%%%%%%%%%%%%%%%%%%%%%%%%%%%%%%%%%%%%%%%%%%%%%
\subsection{Discussion of Results}
Analyzing the results of the present study, researchers highlight the following findings.

Family approval is a very important factor in the choice of a major in higher education. For girls who are considering a major in CS, the significance of this variable was very clear. In the first semester of 2017 we applied the new questionnaire with 8 girls enrolled in the Department of Computer Science at the University of Brasilia , 38\% chose Computer Majors because of the influence of their families. In Brazil this result is also found at the University of Sao Paulo presented in~\cite{saboya_2009}.

The majority of girls believe that there are more boys than girls majoring in CS. This is accordance with research presented in~\cite{Mercier_2006}, about stereotype in computing.

The question of the viability of employment in CS does not seem to be an important factor when choosing a major in CS, although nearly 30\% of the girls responding maybe, showed that they were not clear about the job market in CS. In Brazil this is a sensitive data, because we have an unemployment rate of 13.7\% , as recorded in the quarter ending in March 2017~\footnote{\url{http://www.ibge.gov.br/}}.

In all of the questions on the questionnaire, the frequency with which respondents chose maybe was significant, which indicates the need to intensify actions that disseminate information about the area of CS, with relation to salaries and job opportunities, as well as the question of the gender balance in the field, among others.

The answers reported by the Elementary School students, regardless of the question, responded more frequently in the category maybe, which may indicate that it is better to begin with computer science activities as early as Elementary School, to motivate students to become interested in the area.

Girls are aware that CS needs math. A report from Brazil's National Institute for Educational Studies and Research~\footnote{\url{http://portal.inep.gov.br/web/guest/censo-da-educacao-superior/}} showed that there are low levels of math learning in Middle and High schools. This difficulty in Math can be a important factor that deters girls from enrolling in CS majors.

In general, these findings present insights into the factors that influence girls in choosing, or not, a major in CS, and can indicate directions to be taken in the effort to mitigate the gender disparity. For example, universities could give incentives for activities for disseminating information about the field of CS, which also target the participation of families, in contrast to activities that focus exclusively on High School students.

Another possibility is to increase contact with the field of CS, bringing it into the students homes with the development of relevant applicatives for mobile devices, which are more common in homes than computers, to spark interest. These research results can also be considered iwhen developing public policies, aimed at creating more centers of digital inclusion, given the increasing need for this type of qualified labor.

Therefore, this information provides many insights on the girls of the Federal District. Not only can we have a clearer picture of the Middle and High School girl in the area, we can change the current approaches to address directly what was found to be the most significant attribute. Initially, informal inquiries with new students at the University of Brasília indicated that not many girls knew many details the CS major, and this guided efforts to provide this information to girls in an attempt to increase their interest in CS.

Therefore, this information provides many insights to the motivations of girls in the Federal District. Not only do we have a clearer picture of the Middle and High School girl in the area, we can change the current approaches to directly address what was found to be the most significant attribute. Initially, informal inquiries with new students at the University of Brasília indicated that not many girls knew many details about the CS major, and this has guided efforts to provide this information to girls in an attempt to increase their interest in CS.

Additionally, the analysis raised more interesting questions concerning the questionnaire. Several of the attributes were found to be of less influence when investigating the girls' interest in pursuing a CS degree, so it may be the case that we are not asking the right questions. Other mining approaches might provide different results, and these will be investigated.



%
\section{Conclusion}\label{sec:conclusion}%

In recent years, the gender gap in the field of Computer Science has widened and girls have not shown interest in majoring or becoming professionals. In order to address the issue, we performed a research to better understand this scenario and its causes, investigating the girls' perspectives on a Computer Science major. The knowledge discovered could to guide and support actions that will reduce the disparity by increasing participation of the girls.

Data mining and statistical analysis were applied to the data from a questionnaire done from 2011 to 2014, and results showed that the single most important factor for a Middle or High School girl deciding whether to pursue a degree in a CS major is her family's approval of this choice. Insights on the importance of other factors, such as career opportunities and Math requirements for a major were also obtained. The analysis and the mining process sparked new questions and discussions.

After this research, the project \texttt{Meninas.comp} changed some of its approaches. We created activities targeted exclusively at Middle School girls; provided  lectures specifically about jobs in computing and lectures featuring important women in the field; proposed activities involving Mathematics, Computing and games. We will be watching the results on these actions to see if they will help reduce the gender gap.

Future works include: improving the questionnaire; applying the analysis to data for Middle and High School boys, for comparison, as well as to students who have enrolled in a CS major; and investigating other data mining approaches for knowledge discovery.
%

%% The Appendices part is started with the command \appendix;
%% appendix sections are then done as normal sections
%% \appendix

%% \section{}
%% \label{}
\section*{References}%
\bibliographystyle{elsarticle-num}%
\bibliography{bibliography}%

\end{document}
\endinput
%%
%% End of file `elsarticle-template-num.tex'.

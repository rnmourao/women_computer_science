\section{Introduction}\label{sec:intro}%

%Begin Maristela
In Brazil, the choice of an undergraduate major in the area of Computer Sciences is not among the top choices for girls in high school when contemplating a career. As~\cite{maia_2016} presents, between 2000 and 2013 in Brazil, an average of only 17\% of all graduates in various Computer Science majors were women. This research covered majors in Computer Science, Computer Engineering, and Information Systems, among others. Particularly, in the Federal District, at the University of Brasília, which currently has approximately 30,000 students enrolled in undergraduate programs, the reality is even worse, where in the past 10 years, according to only 10\% were women~\cite{couto_2014}.

Responding to the low incidence of women in Computer Science majors, recently researchers have given much thought about how to improve this scenario and proposed strategies to encourage girls to pursue a profession in the Computer Sciences~\cite{cohoon_2002,couto_2014,gurer_2002,maia_2016}. Brazil and other countries have developed initiatives to debate this issue. Specifically, the Institute of Electrical and Electronics Engineers (IEEE) has a program which address the problem: the IEEE Women in Engineering\footnote{\url{http://wie.ieee.org}} (WIE). The WIE is a major professional and international organization dedicated to promoting women scientists and engineers. Another prominent program in promoting women in the area of Computers is \emph{Girls who Code}\footnote{\url{https://girlswhocode.com}}, which has over 40,000 members and various initiatives to increase the participation of girls in Computer Sciences over various regions in the United States. Another initiative from the United States is the \emph{Grace Hopper Celebration of Women in Computer Sciences}\footnote{\url{http://ghc.anitaborg.org}} event, which is the biggest event worldwide for discussing the theme of women in the field. In 2016 alone 15,000 people from 87 countries participated in the 700 presentations.

In Brazil, since 2007, the Brazilian Society of Computing Conference held the Women in Information and Technology Workshop (WIT), to discuss the theme. Brazilian governmental agencies, such as the Ministry of Science and Technology released calls for submissions of research projects specifically related to the education of girls in Computing or Physical Science majors \cite{cnpq_2017}. Aiming to gather information about the perceptions of high school girls regarding computer science, the Department of Computer Science at the University of Brasília, developed the project, \texttt{Meninas.comp}: computação  também é coisa de menina, Girls.comp: computer science is a girl thing too.

%End Maristela

Between the years of 2011 and 2014, we contacted thousands of people to poll their information, aiming to investigate the relationship between them and the computer, as a tool, and their interest in an undergraduate course in Computer Science. In this work, we used this data to investigate and their affinity with the field, using the Apriori algorithm for searching for association rules on the girls' interest in CS and their background. The goal is to understand such relationships and gain insights on the situation since such knowledge can be useful when addressing the gender difference in the field.

This following sections of this work are organized as follows: Section~\ref{sec:background} presents related work and background information on this approach, Section~\ref{sec:mining} describes details of the data mining applied, Section~\ref{sec:results} provides our experimental results and our finds and Section~\ref{sec:conclusion} presents concluding remarks.

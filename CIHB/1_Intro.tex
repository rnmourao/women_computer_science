\section{Introduction}\label{sec:intro}%
In the years between 2000 and 2013, women accounted for an average of only 17\% of all graduates in various Computer Science (CS) related majors in Brazil~\cite{maia_2016}. This situation is particularly distressing at one of Brazil's largest universities, the capital's University of Bras\'{i}lia (UnB), which has over 30,000 students enrolled in its undergraduate programs. There, over the last ten years, women have accounted for only 10\% of all graduates in CS majors UnB~\cite{couto_2014}.

Responding to these low rates in the number of women in CS courses, researchers have recently focused on how to improve this scenario, and have proposed strategies to encourage girls to pursue a profession in Computer Sciences~\cite{maia_2016,couto_2014,cohoon_2002,gurer_2002}. Several countries have developed initiatives to debate this issue, such as IEEE's \emph{Women in Engineering}\footnote{\url{http://wie.ieee.org}}, ACM's \emph{ACM-W}\footnote{\url{http://women.acm.org/}} and \emph{Grace Hopper Celebration of Women in Computer Sciences}\footnote{\url{http://ghc.anitaborg.org}}, the \emph{Girls Who Code}\footnote{\url{https://girlswhocode.com}} nonprofit organization, among others, all aiming to support and increase the number of women in CS.

The Brazilian Society of Computing has the \emph{Meninas Digitais} project\footnote{\url{http://meninas.sbc.org.br}}, originated from the \emph{Women in Information and Technology} workshops held yearly since 2007 at the country's largest CS conference. The event was created to discuss this theme and Brazilian governmental agencies, such as the Ministry of Science and Technology, have been specifically calling for research projects related to the education of girls in STEM majors~\cite{cnpq_2017}.

Aiming to gather information on high school girls' perceptions of the field, UnB's CS Department started the \texttt{Meninas.comp} project\footnote{\url{https://facebook.com/meninas.comp}}, with the ``Computer Science is a girl's thing too'' motto. This project's goals are to provide qualified information about the computer profession to high school students, to encourage the discussion about the lack of women in CS, to gather data about the process girls go through when choosing their profession, and to promote experimentation with computational activities.

One of the main challenges is understanding why girls do not want a career in CS. Studies show that from the 1980s on, their interest in STEM careers decreased~\cite{Abbate_2012}, and there is a perception of a male image of computer science in general~\cite{Mercier_2006}. In Brazil, there is no current research addressing the cause of so few women in computing so \texttt{Meninas.comp} elaborated a questionnaire of 14 questions about a career in CS to investigate. We applied it between the years of 2011 and 2014,  polling 3707 girls, to examine their relationship with the computer as a tool and their interest in an undergraduate course in CS.

The resulting data was analyzed to examine the girls' affinity with the field, using the Apriori algorithm for searching for the association rules on their interest in CS and their background. The goal is to understand these relationships and gain insights on the gender issue.

The rest of this work is organized as follows: Section~\ref{sec:related} presents related works, Section~\ref{sec:data} provides details on the survey, Section~\ref{sec:perception} describes experimental results and findings, and concluding remarks are given in Section~\ref{sec:conclusion}.

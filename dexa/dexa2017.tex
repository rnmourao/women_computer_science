\documentclass{llncs}

%%%%%%%%%%%%%%%%%%%%%%%%%%%%%%%%%%%%%%%%%%%%%%
\usepackage[utf8]{inputenc}%                 %
\usepackage[T1]{fontenc}%                    %
\usepackage{lmodern}%                        %
\usepackage{graphicx}%                        %
                                             %
\usepackage{xcolor}%                         %
\newcommand{\gnramos}[1]{\textcolor{red}{#1}}%
%%%%%%%%%%%%%%%%%%%%%%%%%%%%%%%%%%%%%%%%%%%%%%

\begin{document}%
\title{XXXXX analysis of middle and high school girls' perspective of the in Brazil's Distrito Federal}%
%
\author{Roberto N. Mourão\inst{1}%
\and Guilherme N. Ramos%
\and Maristela}%
%
\institute{University of Brasília, Brasília - DF, Brazil,\\
\email{mholanda@unb.br},\\%
WWW home page: \texttt{http://cic.unb.br}}%

\maketitle%

\begin{abstract}%
% The abstract should summarize the contents of the paper
% using at least 70 and at most 150 words. It will be set in 9-point
% font size and be inset 1.0 cm from the right and left margins.
% There will be two blank lines before and after the Abstract. \dots
In the last years, many reflections arose about the low rates of female admissions in Computer Science undergraduate programs. Between the years of 2012 and 2014, girls at Middle and High School at Distrito Federal (N = 1707) answered a questionnaire related to their perceptions about Computer Science and their degree of contact with computers, in order to find a relationship between these perceptions and habits, and the intention to applying for a Computer Science undergraduate course. Results from two-way ANOVA indicated that girls who have family approval to applying for a Computer Science course and use LAN Houses have a higher probability to apply for a Computer Science's program.

\keywords{computer science,gender,ANOVA}%
\end{abstract}%

\section{Introduction}\label{sec:intro}%

%Begin Maristela
In Brazil, the choice of an undergraduate major in the area of Computer Sciences is not among the top choices for girls in high school when contemplating a career. As \cite{maia_2016} presents, between 2000 and 2013 in Brazil, an average of only 17\% of all graduates in various Computer Science majors were women. This research covered majors in Computer Science, Computer Engineering, and Information Systems, among others. Particularly, in the Federal District, at the University of Brasilia, which currently has approximately 30,000 students enrolled in undergraduate programs, the reality is even worse, where in the past 10 years, according to \cite{couto_2014} only 10\% were women.

	Responding to the low incidence of women in Computer Science majors, recently researchers have given much thought about how to improve this scenario and proposed strategies to encourage girls to pursue a profession in the Computer Sciences   \cite{cohoon_2002} \cite{couto_2014}  \cite{gurer_2002}  \cite{maia_2016}. Brazil and other countries have developed initiatives to debate this issue. Specifically, the Institute of Electrical and Electronics Engineers (IEEE) has a program which address the problem: the IEEE Women in Engineering (WIE) \cite{wie2017}. The WIE is a major professional and international organization dedicated to promoting women scientists and engineers. Another prominent program in promoting women in the area of Computers is ?Girls who Code? \cite{girlsWC_2017}, which has over 40,000 members and various initiatives to increase the participation of girls in Computer Sciences over various regions in the United States. Another initiative from the United States is the ?Grace Hopper Celebration of Women in Computer Sciences? event, which is the biggest event worldwide for discussing the theme of women in the field. In 2016 alone 15,000 people from 87 countries participated in the 700 presentations \cite{GHC_2017}.

	In Brazil, since 2007, the Brazilian Society of Computing Conference held the Women in Information and Technology Workshop (WIT), to discuss the theme. Brazilian governmental agencies, such as the Ministry of Science and Technology released calls for submissions of research projects specifically related to the education of girls in Computing or Physical Science majors \cite{cnpq_2017}. Aiming to gather information about the perceptions of high school girls regarding computer science, the Department of Computer Science at the University of Brasilia, developed the project, Meninas.comp: computação  tambem e coisa de menina, Girls.comp: computer science is a girl thing too.

%End Maristela

\gnramos{Falar da análise realizada (Roberto e Guilherme)}%

Between the years of 2012 and 2014, we contacted thousands of girls in Middle and High School to investigate the relationship between their intention to apply for an undergraduate course in Computer Science and their affinity with the field and computing tools. We used the Apriori algorithm on the collected data, searching for interesting association rules and insights on the girls' interest in CS and their background.

This following sections of this work are organized as follows: Section~\ref{sec:background} presents related work and background information on this approach, Section~\ref{sec:mining} describes details of the data mining applied, Section~\ref{sec:results} provides our experimental results and our finds and Section~\ref{sec:conclusion} presents concluding remarks.
%
\section{Background \& Related Works}\label{sec:background}%

\subsection{Girls in Computer Science}\label{subsec:background:girls}%
2. Meninas e a Computação (Maristela)

Middle School: Ensino Fundamental II
High School: Ensino Médio

\subsection{Data Analysis}\label{subsec:background:data}%

Data analysis includes, among other things, procedures for analyzing data and techniques for interpreting their results~\cite{Tukey1962}.

\gnramos{Breve relação das técnicas utilizadas}%
ANOVA ~\cite{Hastie2009}


Frequently, data may be usefully and accurately treated by the analysis of variance (through separation of the variance attributes to a group of causes from that attributed to other groups)~\cite{Fisher1934}. This \emph{analysis of variance} (ANOVA)can be achieved through several approaches, for example through a hypothesis test comparing the two models, considering the null hypothesis that both models fit the data equally well in contrast to  the full model is superior~\cite{James2013}.

\gnramos{Trabalhos correlacionados}%


%
\section{Data Mining}\label{sec:mining}%
%3. Teoria sobre a análise dos Dados (Roberto e Guilherme).%

Data mining can be defined as the process of discovering insightful patterns and predictive models from data~\cite{Zaki2014}, trying to make sense of usually large amounts of data in some domain~\cite{Cios2007}. \gnramos{In this work, the main interest is the gender gap in the fields of the ``hard sciences'', so we attempt to understand the profiles of women intending to enroll in undergraduate studies, specially those interested in Computer Science.}

From 2012 to 2014, 1709 people were polled, responding a questionnaire\footnote{\url{https://github.com/rnmourao/women\_computer\_science/blob/master/data/questionnaire.pdf}}. The idea was to obtain information on their perceptions of their future undergraduate studies.

\subsection{Poll}\label{sec:mining:poll}%
The poll had 18 questions, decribed below:

\begin{enumerate}
	\item Gender:
		\begin{enumerate}
			\item Female
			\item Male
		\end{enumerate}
	\item Educational Stage:
		\begin{enumerate}
			\item Middle School
			\item High School (10th Grade)
			\item High School (11th Grade)
			\item High School (12th Grade)	
			\item Adult Education Program
			\item College		
		\end{enumerate}
	\item Which is your Field of Interest?
		\begin{itemize}
			\item Biology-Health Sciences 
			\item Human Sciences
			\item Exact Sciences	
		\end{itemize}
	\item Are you intend to apply for a Computer Science course?
		\begin{itemize}
			\item Yes
			\item No
			\item Maybe
		\end{itemize}
	\item Mark all places where you use computers:
		\begin{itemize}
			\item Home
			\item Relatives' House
			\item Friends's House
			\item School
			\item Work
			\item Lan House
			\item Library
			\item Digital Inclusion Centre
		\end{itemize}
	\item Mark all softwares or activities you use or do with  a computer:
		\begin{itemize}
			\item Text Editor (Microsoft Word, etc)
			\item Image Editor
			\item Spreadsheet
			\item Database
			\item Web Browser (Search Engines, News, etc)
			\item Social Networks (Facebook, Orkut, etc)
			\item E-mail
			\item Games
			\item Create Web Pages
			\item Create Softwares
			\item Other Softwares
		\end{itemize}
	\item Does a Computer Science course only teaches how to use softwares?
		\begin{itemize}
			\item Yes
			\item No
			\item Maybe
		\end{itemize}
	\item Does a Computer Science course uses easy Math?
		\begin{itemize}
			\item Yes
			\item No
			\item Maybe
		\end{itemize}	
	\item Does the majority of Computer Science's students are male?
		\begin{itemize}
			\item Yes
			\item No
			\item Maybe
		\end{itemize}
	\item Is it necessary to know how to use computers to apply for a Computer Science course?
		\begin{itemize}
			\item Yes
			\item No
			\item Maybe
		\end{itemize}							
	\item Is it necessary to graduate in Computer Science to work in the area?
		\begin{itemize}
			\item Yes
			\item No
			\item Maybe
		\end{itemize}		
	\item Would your family like you to apply for a Computer Science course?
		\begin{itemize}
			\item Yes
			\item No
			\item Maybe
		\end{itemize}				
	\item It is difficult to get a job after undergraduate in Computer Science?
		\begin{itemize}
			\item Yes
			\item No
			\item Maybe
		\end{itemize}		
	\item Who works with Computer Science has few hours of leisure?
		\begin{itemize}
			\item Yes
			\item No
			\item Maybe
		\end{itemize}	
	\item Does working with Computer Science allow you to exercise creativity?
		\begin{itemize}
			\item Yes
			\item No
			\item Maybe
		\end{itemize}
	\item Does working with Computer Science brings prestige?
		\begin{itemize}
			\item Yes
			\item No
			\item Maybe
		\end{itemize}	
	\item Does working with Computer Science allow you to earn a good salary?
		\begin{itemize}
			\item Yes
			\item No
			\item Maybe
		\end{itemize}
	\item Does working with Computer Science allow you to work in other fields?
		\begin{itemize}
			\item Yes
			\item No
			\item Maybe
		\end{itemize}									
\end{enumerate}

The questions were translated into 36 variables, as following: Gender, Educational.Stage, Field.Interest, Interest.CS, Computer.Home, Computer.Relatives, Computer.Friends, Computer.School, Computer.Work, Computer.Lan.House, Computer.Library, Computer.Digital.Inclusion.Centre, Use.Text.Editor, Use.Image.Editor, Use.Spreadsheet, Use.Database, Use.Internet, Use.Social.Network, Use.Email, Use.Games, Create.Web.Pages, Create.Softwares, Use.Other.Softwares, Only.Teaches.Software, Has.Low.Math, Man.Majority, Need.Know.Computer, Need.Higher.Education, Family.Approval, Low.Employability, Low.Leisure, Use.Creativity, Bring.Prestige, Good.Salary, Interdisciplinarity, and Year.

\gnramos{descrição das perguntas, motivações, tipos de respostas (categoricas, numericas...)}

\subsection{Statistical Analysis}\label{sec:mining:stat}%
The usual pre-processing tasks were quickly done; all questionnaires were created and processed by CESPE\footnote{\url{http://www.cespe.unb.br/}}, a specialized research center, and the results given in simple spreadsheets. The data for all years was consolidated in a single spreadsheet\footnote{\url{https://github.com/rnmourao/women\_computer\_science/blob/master/data/raw.xlsx}}
which was then processed in the \texttt{R} programming language.

The script cleans up the data (empty columns, whitespaces, etc.) and begins to process it for analysis. The first step is to consider only the data for the 1709 female respondents. The data is distributed throughout the years as illustrated in Figure~\ref{fig:RespondentsPerYear}. \gnramos{Descrição desta informação}.

\begin{figure}%
\includegraphics[width=\textwidth]{img/RespondentsPerYear}%
\caption{Number of respondents per year.}%
\label{fig:RespondentsPerYear}%
\end{figure}%

Figure~\ref{fig:EducationalStage} shows how the respondents were distributed by educational stage, indicates that \gnramos{qual a relevância desta informação?},

\begin{figure}%
\includegraphics[width=\textwidth]{img/EducationalStage}%
\caption{Educational Stages of respondents.}%
\label{fig:EducationalStage}%
\end{figure}%

Figure~\ref{fig:FieldOfInterest} shows the respondents' interest in applying for a Computer Science course. This results indicate that $31\%$  \emph{have interest}, $28\%$  \emph{have no interest}, and $41\%$  \emph{have doubt}.

\begin{figure}%
\includegraphics[width=\textwidth]{img/FieldOfInterest}%
\caption{Respondents' fields of interest.}%
\label{fig:FieldOfInterest}%
\end{figure}%

Figure~\ref{fig:InterestInCS.pdf} shows the respondents' interest in different fields of study. This results indicate that, throughout the years, the percentages for each choice remains more or less the same with an average of $41\%$ for \emph{Biology-Health Sciences}, $22\%$ for \emph{Exact Sciences} and $33\%$ for \emph{Human Sciences}.

\begin{figure}%
\includegraphics[width=\textwidth]{img/InterestInCS.pdf}%
\caption{Respondents interested in Computer Science.}%
\label{fig:InterestInCS.pdf}%
\end{figure}%

% ANOVA %%%%%%%%%%%
As described earlier, the data consisted of a group of 1,709 questionnaires with 18 questions each. There were 3,707 questionnaires, but some were excluded for many reasons: some were answered by men, others were answered by College or Adult Education Program students. None of these were adequate for the study. The questionnaires from 2011 were excluded because the poll was conducted differently of the others years.

The \emph{CS.Interest} attribute, related to the intention of the respondent for applying to a Computer Science Undergraduate Course, was used to create a numeric variable, named as \emph{CS.Choice}. When there was `No' in CS.Interest, CS.Choice was filled with the value -1; when `Maybe', CS.Choice got 0, and when CS.Interest was `Yes', CS.Choice received the value 1. In this study, the attribute \emph{CS.Choice} was used as response variable~\cite{moore2009practice}. The others 35 attributes were used as explanatory variables, turning in a treatment each possible answer for each variable.

The grand mean, i.e., the general mean for the attribute CS.Choice was 0.03.

The next step was to perform a One-Way ANOVA with the treatments of all attributes, removing one by one, based on the treatments' significance~\cite{Chambers1990}. Since the data was naturally unbalanced, the ANOVA used the `Type III' Sum of Squares. 

A series of One-Way ANOVAs was performed to compare the effect of each remaining attribute on the students' intention to applying for a Computer Science Course. A Tukey HSD test was used for the cases when there was more than two treatments.
 
Then, it was executed a Two-Way ANOVA to evaluate the interactions between the attributes' treatments.

After analyze each treatment, there was selected the treatment with significance, and higher mean for the CS.Choice attribute.



%
\section{Experimental Results}\label{sec:results}%

4. Os resultados coletados (Roberto e Guilherme)

\subsection{Discussion}\label{subsec:discussion}%
Análise dos Resultados (Maristela, Roberto e Guilherme%
\section{Conclusion}\label{sec:conclusion}%

In recent years, the gender gap in the field of Computer Science has widened and girls have not shown interest in majoring or becoming professionals. In order to address the issue, we performed a research to better understand this scenario and its causes, investigating the girls' perspectives on a Computer Science major. The knowledge discovered could to guide and support actions that will reduce the disparity by increasing participation of the girls.

Data mining and statistical analysis were applied to the data from a questionnaire done from 2011 to 2014, and results showed that the single most important factor for a Middle or High School girl deciding whether to pursue a degree in a CS major is her family's approval of this choice. Insights on the importance of other factors, such as career opportunities and Math requirements for a major were also obtained. The analysis and the mining process sparked new questions and discussions.

After this research, the project \texttt{Meninas.comp} changed some of its approaches. We created activities targeted exclusively at Middle School girls; provided  lectures specifically about jobs in computing and lectures featuring important women in the field; proposed activities involving Mathematics, Computing and games. We will be watching the results on these actions to see if they will help reduce the gender gap.

Future works include: improving the questionnaire; applying the analysis to data for Middle and High School boys, for comparison, as well as to students who have enrolled in a CS major; and investigating other data mining approaches for knowledge discovery.
%

\bibliographystyle{splncs03}%
\bibliography{bibliography}%
\end{document}%

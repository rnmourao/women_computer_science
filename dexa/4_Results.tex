\section{Experimental Results}\label{sec:results}%

% Family.Approval, Use.Creativity %

The association rule mining resulted in 1,171 rules involving the students' interest in Computer Science. Rules with lift less than 1.8 were excluded. Figure \ref{fig:apriori} shows rules' left-hand side (lhs), rules' right-hand side (rhs), support, confidence and lift metrics.

\begin{figure}%
\includegraphics[width=\textwidth]{img/apriori.pdf}%
\caption{Rules with higher confidence, ordered by lift.}%
\label{fig:apriori}%
\end{figure}%

The rules found had very close levels of confidence, between 0.56 and 0.59. The lift metric was similar, with values between 1.8 and 1.9. Figure \ref{fig:plot.apriori} shows in a chart this conclusions.

\begin{figure}%
\includegraphics[width=\textwidth]{img/{plot.apriori}.pdf}%
\caption{Rules with higher confidence, ordered by lift.}%
\label{fig:plot.apriori}%
\end{figure}%

% Family.Approval %
A general result was the presence of \emph{Family.Approval=Yes} item in all rules. The values found in \emph{Family.Approval} attribute were three: \emph{Yes}, indicating the student believes that her family may approve her application for a Computer Science course; \emph{No}, stating that the student believes in family disapproval; and  \emph{Maybe} indicates that the students have doubts about her family's opinion. Figure \ref{fig:plot.Family.Approval} shows the number of respondents by answer involving \emph{Family.Approval} and \emph{CS.Interest} attributes.

Support metrics found in Figure \ref{fig:apriori} are varying from 10 to 12\%. This percentage is significant when compared to the total of respondents interested in CS (31\%).

% Confidence varied from 50 to 59\%. %

% Lift varies from 1.8 to 1.9. This metric %


The first rule on figure \ref{fig:apriori} had the highest confidence level. The confidence metric, in this case,  states that 59\% of times a respondent believes that has family's approval and doesn't use computer at home, then she will have an interest in CS. The 1.9 lift indicates that there are 90\% of chance that the family's approval, the lack of use of computer at home and the interest to apply for a CS course are correlated. 

The second rule on figure \ref{fig:apriori} states that 58\% of times a respondent believes that has family's approval and believes that CS uses creativity, then she will have an interest in CS. The 1.9 lift indicates that there are 90\% of chance that the family's approval, the perception of creativity in CS and the interest to apply for a CS course are correlated.

The third rule 

The second rule in figure \ref{fig:apriori} indicates that family's approval is, per si, an important rule. Confidence and lift are only a bit smaller than their first rule pairs. The figure \ref{fig:plot.Family.Approval} has the number of respondents per girl's guess of family's opinion and own answer about interest in CS.

\begin{figure}%
\includegraphics[width=\textwidth]{img/{plot.Family.Approval}.pdf}%
\caption{Amount of respondents by answer in Family.Approval and in CS.Interest.}%
\label{fig:plot.Family.Approval}%
\end{figure}%

While the \emph{Family.Approval} attribute appears as an isolated rule in figure \ref{fig:apriori}, it is noticeable that this does not happen for \emph{Use.Creativity}. The absence of such rule indicates that creativity alone is not a strong stimulus to respondents decide for Computer Science. Figure \ref{fig:plot.Use.Creativity} shows the number of respondents per girl's opinion about the use of creativity in CS and the interest in applying for a Computer Science course.

\begin{figure}%
\includegraphics[width=\textwidth]{img/{plot.Use.Creativity}.pdf}%
\caption{Amount of respondents by answer in Use.Creativity and in CS.Interest.}%
\label{fig:plot.Use.Creativity}%
\end{figure}%
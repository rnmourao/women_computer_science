\section{Experimental Results}\label{sec:results}%
In order to try to understand the profiles of girls intending to enroll in undergraduate courses, especially those interested in Computer Science, we applied Apriori's association rule algorithm to the data, with the minimum confidence level equal to 50\%, and with a maximum number of 3 items in an itemset. A filter was applied to select only the rules involving the variable \emph{Would.Enroll.In.CS}, which indicates the respondent is interested in pursuing a CS degree, on the rules' right-hand sides. The rules were analyzed considering \emph{support}, \emph{confidence} and \emph{lift} metrics.

The association rule mining resulted in 1,171 rules involving the students' interest in Computer Science, their details are presented in Figure~\ref{fig:plot.apriori}.

\begin{figure}%
\centering
\includegraphics[scale=0.5]{{img/plot.apriori}.pdf}%
\caption{Rules with higher confidence, ordered by lift.}%
\label{fig:plot.apriori}%
\end{figure}%

This set is too large for visual investigation, but it is clear that there are few rules with higher lift (stronger red color), which are the ones we are interested in. Thus we looked into the subset of rules with lift of at least 50\%, \gnramos{inserir a justificativa disto}, which consisted in the more manageable set of ten rules shown in Figure~\ref{fig:apriori}.

\begin{figure}%
\includegraphics[width=\textwidth]{img/apriori}%
\caption{Rules with higher confidence, ordered by lift.}%
\label{fig:apriori}%
\end{figure}%

% Family.Approves.CS.Major %
The clearest information from this result was the presence of the \mbox{\emph{Family.Approves.CS.Major}} attribute in all rules. This question was stated as ``\emph{Would your family approve if you applied for a Computer Science major?}'', with three possible answers: \emph{Yes} and \emph{No}, with obvious meaning, and \emph{Maybe}, which indicates that the students have doubts about their families' opinions. Figure~\ref{fig:plot.Family.Approves.CS.Major} shows the number of respondents by answer involving \mbox{\emph{Family.Approves.CS.Major}} and \mbox{\emph{Would.Enroll.In.CS}} attributes.

\begin{figure}%
\includegraphics[width=\textwidth]{img/{plot.Family.Approves.CS.Major}.pdf}%
\caption{Respondents by answer for family approval and interest in a CS major.}%
\label{fig:plot.Family.Approves.CS.Major}%
\end{figure}%

The rule with highest lift (first in Figure~\ref{fig:apriori}) also had the highest confidence level; so we can say that \gnramos{there is 59\% chance for the rule ``\emph{if the respondent believes that she has the family's approval and she does not use a computer at work, then she would enroll in a CS major}''.} The value of 1.9 for lift indicates that there is a 90\% of chance that the antecedents and the consequent for this rule are correlated, which is a very intriguing discovery. However, there seems to be no strong correlation between using computers at work and an interest in a CS major, as shown in Figure~\ref{fig:plot.Uses.Computer.At.Work}. This disparity may come from the fact that many many students work, so the contact with a computer is an everyday happening which would not influence their choice of higher education. A survey by the Brazilian Institute of Geography and Statistics (IBGE) in 2007\footnote{\url{http://goo.gl/R6yyJw}} stated that 22\% of teenage girls in Brazil are part of the country's work force.

\begin{figure}%
\includegraphics[width=\textwidth]{img/{plot.Uses.Computer.At.Work}.pdf}%
\caption{Respondents by answer for use of computer at work and interest in CS.}%
\label{fig:plot.Uses.Computer.At.Work}%
\end{figure}%

The rule with second highest lift means that 58\% of times ``\emph{if the respondent believes that she has the family's approval and she believes Computer Science fosters creativity, then she would enroll in a CS major}''. Again, there is a very strong correlation between the antecedents and the consequent.

\gnramos{Não está claro o problema da regra e o fato de ``usar computador no trabalho''. Comentei o texto para discutirmos melhor}
However, there seems to be no great difference between students' opinions on creativity and their interest in a CS major, as illustrated in Figure~\ref{fig:plot.CS.Fosters.Creativity}. Thus the \emph{CS.Fosters.Creativity} and the \emph{Uses.Computer.At.Work} attributes have no significance compared to \emph{Family.Approves.CS.Major}.

\begin{figure}%
\includegraphics[width=\textwidth]{img/{plot.CS.Fosters.Creativity}.pdf}%
\caption{Amount of respondents by answer in CS.Fosters.Creativity and in Would.Enroll.In.CS.}%
\label{fig:plot.CS.Fosters.Creativity}%
\end{figure}%

The third rule in figure \ref{fig:apriori} indicates that family's approval is, per si, an important rule. Confidence and lift are only a bit smaller than their first rule pairs. The figure \ref{fig:freq.Family.Approves.CS.Major} has the number of respondents per girl's guess of family's opinion and own answer about interest in CS.

\begin{figure}%
\centering
\includegraphics[scale=0.8]{img/{freq.Family.Approves.CS.Major}.pdf}%
\caption{Amount of respondents by answer in Family.Approves.CS.Major and in Would.Enroll.In.CS.}%
\label{fig:freq.Family.Approves.CS.Major}%
\end{figure}%

Other attributes shown in Figure \ref{fig:apriori} have low significance, like Uses.Computer.At.Work and CS.Fosters.Creativity, as shown in figures \ref{fig:plot.Uses.Computer.At.Digital.Inclusion.Center}~\ref{fig:plot.Uses.Computer.At.Home}~\ref{fig:plot.Uses.Computer.At.Lan.House}~\ref{fig:plot.Has.Used.Internet}~\ref{fig:plot.Has.Used.Social.Network}~\ref{fig:plot.Has.Used.Text.Editor}. Consequently, all these results indicate that the family's opinion is the most important factor to these girls, in order to decide to apply (or not) for a CS course.

\begin{figure}%
\includegraphics[width=\textwidth]{img/{plot.Uses.Computer.At.Digital.Inclusion.Center}.pdf}%
\caption{Amount of respondents by answer in Uses.Computer.At.Digital.Inclusion.Center and in Would.Enroll.In.CS.}%
\label{fig:plot.Uses.Computer.At.Digital.Inclusion.Center}%
\end{figure}%

\begin{figure}%
\includegraphics[width=\textwidth]{img/{plot.Uses.Computer.At.Home}.pdf}%
\caption{Amount of respondents by answer in Uses.Computer.At.Home and in Would.Enroll.In.CS.}%
\label{fig:plot.Uses.Computer.At.Home}%
\end{figure}%

\begin{figure}%
\includegraphics[width=\textwidth]{img/{plot.Uses.Computer.At.Lan.House}.pdf}%
\caption{Amount of respondents by answer in Uses.Computer.At.Lan.House and in Would.Enroll.In.CS.}%
\label{fig:plot.Uses.Computer.At.Lan.House}%
\end{figure}%

\begin{figure}%
\includegraphics[width=\textwidth]{img/{plot.Has.Used.Internet}.pdf}%
\caption{Amount of respondents by answer in Has.Used.Internet and in Would.Enroll.In.CS.}%
\label{fig:plot.Has.Used.Internet}%
\end{figure}%

\begin{figure}%
\includegraphics[width=\textwidth]{img/{plot.Has.Used.Social.Network}.pdf}%
\caption{Amount of respondents by answer in Use.Social.Network and in Would.Enroll.In.CS.}%
\label{fig:plot.Has.Used.Social.Network}%
\end{figure}%

\begin{figure}%
\includegraphics[width=\textwidth]{img/{plot.Has.Used.Text.Editor}.pdf}%
\caption{Amount of respondents by answer in Use.Text.Editor and in Would.Enroll.In.CS.}%
\label{fig:plot.Has.Used.Text.Editor}%
\end{figure}%



\section{Experimental Results}\label{sec:results}%

% `Family.Approval', `Computer.Lan.House' %

As result of the One-Way and Two-Way ANOVA, figures ~\ref{fig:significant.main.effects} and ~\ref{fig:significant.interactions}, respectively, show the treatments and the interactions which have treatments with higher means for the attribute `CS.Choice'. The highest mean (.60) encountered was originated from the interaction between the attributes `Family.Approval' and `Computer.Lan.House'.

\begin{figure}%
\includegraphics[width=\textwidth]{img/{significant.main.effects}.pdf}%
\caption{Attributes with most significant treatments, ordered by means.}%
\label{fig:significant.main.effects}%
\end{figure}%

\begin{figure}%
\includegraphics[width=\textwidth]{img/{significant.interactions}.pdf}%
\caption{Interactions with most significant treatments, ordered by means.}%
\label{fig:significant.interactions}%
\end{figure}%

The results of ANOVA for each treatment and interaction concerning the two attributes are explained below.

% Family.Approval %
The treatments extracted from the attribute Family.Approval were three: `Yes', indicating that the student believes that her family may approve her application for a Computer Science Course; `No', indicating that the student believes in family disapproval; and  `Maybe' indicates that the students have doubts about her family's opinion. Figure ~\ref{fig:plot.Family.Approval} shows the number of respondents for each treatment about the interest to apply for Computer Science.

\begin{figure}%
\includegraphics[width=\textwidth]{img/{plot.Family.Approval}.pdf}%
\caption{Amount of respondents by answer in Family.Approval and in CS.Interest.}%
\label{fig:plot.Family.Approval}%
\end{figure}%

An analysis of variance ~\ref{fig:anova.Family.Approval} showed that the effect of the attribute Family.Approval was significant, with F value equal to 68.0 and the p-value less than 0.05.

\begin{figure}%
\includegraphics[width=\textwidth]{img/{anova.Family.Approval}.pdf}%
\caption{One-Way ANOVA for the treatments related to the attribute Family.Approval.}%
\label{fig:anova.Family.Approval}%
\end{figure}%

Post hoc comparisons using the Tukey HSD test ~\ref{fig:tukey.Family.Approval} indicated that the mean score for the perspective of family approval (M = 0.44, SD = 0.72) was significantly different than the perspective of family's disapproval (M = -0.27, SD = 0.75) and different than the doubt about fammily's approval or disapproval (M = 0.04, SD = 0.71).  

\begin{figure}%
\includegraphics[width=\textwidth]{img/{tukey.Family.Approval}.pdf}%
\caption{Tukey test for the treatments related to the attribute Family.Approval.}%
\label{fig:tukey.Family.Approval}%
\end{figure}%

% Computer.Lan.House %
The treatments extracted from the attribute Computer.Lan.House were two: `Yes', indicating that the students use computers in Lan Gaming Houses; `No', indicating that the student doesn't use computers in Lan Gaming Houses. Figure ~\ref{fig:plot.Computer.Lan.House} shows the number of respondents for each treatment about the interest to apply for Computer Science.

\begin{figure}%
\includegraphics[width=\textwidth]{img/{plot.Computer.Lan.House}.pdf}%
\caption{Amount of respondents by answer in Computer.Lan.House and in CS.Interest.}%
\label{fig:plot.Computer.Lan.House}%
\end{figure}%

An analysis of variance ~\ref{fig:anova.Computer.Lan.House} showed that the effect of the attribute Computer.Lan.House was significant, with the F value equal to 61.3 and the p-value less than 0.05.

\begin{figure}%
\includegraphics[width=\textwidth]{img/{anova.Computer.Lan.House}.pdf}%
\caption{One-Way ANOVA for the treatments related to the attribute Computer.Lan.House.}%
\label{fig:anova.Computer.Lan.House}%
\end{figure}%

The analysis indicated that the mean score for the answer `Yes' (M = 0.20, SD = 0.74) was significantly different than the answer `No' (M = -0.03, SD = 0.76) for the attribute Computer.Lan.House.

% Family.Approval x Computer.Lan.House %
The Two-Way ANOVA revealed that the effect of use of computer in Lan Gaming Houses enhances the effect of family's approval.  The F value was equal to 38.9 and the p-value less than 0.05, as shown in figure ~\ref{fig:anova.interaction}. 

\begin{figure}%
\includegraphics[width=\textwidth]{img/{anova.Family.Approval.x.Computer.Lan.House}.pdf}%
\caption{Two-Way ANOVA with interaction for the treatments related to the attributes Family.Approval and  Computer.Lan.House.}%
\label{fig:anova.interaction}%
\end{figure}%

Post hoc comparisons using the Tukey HSD test ~\ref{fig:tukey.interaction} indicated that the mean score for the perspective of family's approval and the use of Lan Houses (M = 0.60, SD = 0.62) was significantly different than the others interactions. The family's approval and the non attendance in Lan Houses, e.g., lowered the mean of CS.Choice attribute (M = 0.35, SD = 0.76), when compared against the single effect of family's approval ~\ref{fig:tukey.Family.Approval}.

\begin{figure}%
\includegraphics[width=\textwidth]{img/{tukey.Family.Approval.x.Computer.Lan.House}.pdf}%
\caption{Tukey test for the treatments related to the attributes Family.Approval and Computer.Lan.House.}%
\label{fig:tukey.interaction}%
\end{figure}%

\subsection{Discussion}\label{subsec:discussion}%

The results of the One-Way ANOVA for the treatments related the family's opinion about the girl's interest in applying to a Computer Science course suggest that the family's aval is very relevant when the girl decides to choose a CS course. In the same way, the results suggest that a family disapproval could make the girl decide not to apply to Computer Science. A course of action to increase the interest for Computer Science could be a series of lectures directed both to parents and daughters, explaining the computer scientist's career.

The One-Way ANOVA suggests that the use of computer in Lan Gaming Houses increases the girls' interest to applying for a Computer Science course. 

Finally, the Two-Way ANOVA suggests that the interaction between these two treatments can increase the girl's interest in Computer Science. Maybe the invite for lectures about computer scientist's carreer could be extended to Lan Gaming Houses participants and her parents. 
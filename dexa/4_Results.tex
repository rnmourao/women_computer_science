\section{Experimental Results}\label{sec:results}%

As described earlier, the data consists of a group of 3,161 questionnaires with 18 questions each. There was 3,707 questionnaires, but some were excluded for many reasons: some where answered by men, others where answered by College or Adult Education Program students. None of these were adequate for the study. 

One of these questions is related to the intention of the respondent for applying to a Computer Science Undergraduate Course. The possible answers were No, Maybe and Yes. The answers for this question were converted to a scale where No received the value -1, Maybe received the value 0, and Yes receive the value 1. The name of this attribute is CS_Choice.

The others questions were converted to another 35 attributes, as explained earlier.

The first step was to remove all questionnaires of male respondents. After this, the attribute Gender was removed. The questionnaires answered by students of College and Adult Education Program were removed, too.

Thus, the number of questionnaires was reduced to 3,161, with 745 students that answered No to intention in applying for a Computer Science Undergraduate Course, 1,238 students that marked Maybe, and 1,178 that would like to take a CS course. The grand mean, i.e., the genereal mean for the attribute CS_Choice was 0.14. This number shows a general uncertainty about the decision for application on a Computer Science Course. 

The next step was to perform a One-Way ANOVA to compare the effect of each attribute on the students' intention of applying for a Computer Science Course.  

% Bring_Prestige %
The treatments extracted from the attribute Bring_Prestige are two: `Yes', indicating that the student believes that taking a Computer Science course would bring to her prestige; `No', indicating that the student believes that taking such course wouldn't bring prestige to her; and `Maybe', indicating that the student has doubts about the prestige of a career as computer scientist. Figure ~\ref{fig:PLOT_Bring_Prestige} shows the quantity of respondents for each treatment about the interest to apply for Computer Science.

\begin{figure}%
\includegraphics[width=\textwidth]{img/plot_Bring_Prestige}%
\caption{Amount of respondents by answer in Bring_Prestige and in CS_Interest.}%
\label{fig:plot_Bring_Prestige}%
\end{figure}%

An analysis of variance ~\ref{fig:ANOVA_Bring_Prestige} showed that the effect the attribute Bring_Prestige was significant, F(3, 3053) = 42.65, p = .000.

\begin{figure}%
\includegraphics[width=\textwidth]{img/anova_Bring_Prestige}%
\caption{One-Way ANOVA for the treatments related to the attribute Bring_Prestige.}%
\label{fig:ANOVA_Bring_Prestige}%
\end{figure}%

Post hoc comparisons using the Tukey-Kramer test ~\ref{fig:ANOVA_Bring_Prestige} indicated that the mean score for the answer `Yes' (M = 0.20, SD = 0.75) was significantly different than answer `Maybe' (M = 0.01, SD = 0.76). However, the answer `No'  (M = 0.09, SD = 0.86) did not significantly differ from the answers `Yes' and `Maybe'.  

\begin{figure}%
\includegraphics[width=\textwidth]{img/tukey_Bring_Prestige}%
\caption{Tukey test for the treatments related to the attribute Bring_Prestige.}%
\label{fig:TUKEY_Bring_Prestige}%
\end{figure}%
% \Bring_Prestige %

% Computer_Digital_Inclusion_Centre %
The treatments extracted from the attribute Computer_Digital_Inclusion_Centre are two: `Yes', indicating that the student goes to digital inclusion centres to access computers; `No', indicating that the student doesn't go to digital inclusion centres. Figure ~\ref{fig:PLOT_Computer_Digital_Inclusion_Centre} shows the quantity of respondents for each treatment about the interest to apply for Computer Science.

\begin{figure}%
\includegraphics[width=\textwidth]{img/plot_Computer_Digital_Inclusion_Centre}%
\caption{Amount of respondents by answer in Computer_Digital_Inclusion_Centre and in CS_Interest.}%
\label{fig:plot_Computer_Digital_Inclusion_Centre}%
\end{figure}%

An analysis of variance ~\ref{fig:ANOVA_Computer_Digital_Inclusion_Centre} showed that the effect of the attribute Computer_Digital_Inclusion_Centre was significant, F(2, 3159) = 61.28, p = .000.

\begin{figure}%
\includegraphics[width=\textwidth]{img/anova_Computer_Digital_Inclusion_Centre}%
\caption{One-Way ANOVA for the treatments related to the attribute Computer_Digital_Inclusion_Centre.}%
\label{fig:ANOVA_Computer_Digital_Inclusion_Centre}%
\end{figure}%

The analysis indicated that the mean score for the answer `Yes' (M = 0.37, SD = 0.74) was significantly different than the answer `No' (M = 0.12, SD = 0.77) for the attribute Computer_Digital_Inclusion_Centre.
% \Computer_Digital_Inclusion_Centre %

% Computer_Friends %
The treatments extracted from the attribute Computer_Friends are two: `Yes', indicating that the student uses computer in her friends' house; `No', indicating that the student doesn't use computer in her friends' house. Figure ~\ref{fig:PLOT_Computer_Friends} shows the quantity of respondents for each treatment about the interest to apply for Computer Science.

\begin{figure}%
\includegraphics[width=\textwidth]{img/plot_Computer_Friends}%
\caption{Amount of respondents by answer in Computer_Friends and in CS_Interest.}%
\label{fig:plot_Computer_Friends}%
\end{figure}%

An analysis of variance ~\ref{fig:ANOVA_Computer_Friends} showed that the effect of the attribute Computer_Friends was significant, F(2, 3159) = 53.41, p = .000.

\begin{figure}%
\includegraphics[width=\textwidth]{img/anova_Computer_Friends}%
\caption{One-Way ANOVA for the treatments related to the attribute Computer_Friends.}%
\label{fig:ANOVA_Computer_Friends}%
\end{figure}%

The analysis indicated that the mean score for the answer `Yes' (M = 0.11, SD = 0.76) was significantly different than the answer `No' (M = 0.19, SD = 0.79) for the attribute Computer_Friends.
% \Computer_Friends %

% Family_Approval %
The treatments extracted from the attribute Family_Approval are three: `Yes', indicating that the student believes that her family may approve her application for a Computer Science Course; `No', indicating that the student believes in family disapproval; and  `Maybe' indicates that the students has doubts about her family's opinion. Figure ~\ref{fig:PLOT_Family_Approval} shows the quantity of respondents for each treatment about the interest to apply for Computer Science.

\begin{figure}%
\includegraphics[width=\textwidth]{img/plot_Family_Approval}%
\caption{Amount of respondents by answer in Family_Approval and in CS_Interest.}%
\label{fig:plot_Family_Approval}%
\end{figure}%

An analysis of variance ~\ref{fig:ANOVA_Family_Approval} showed that the effect of the attribute Family_Approval was significant, F(3, 3070) = 160.68, p = .000.

\begin{figure}%
\includegraphics[width=\textwidth]{img/anova_Family_Approval}%
\caption{One-Way ANOVA for the treatments related to the attribute Family_Approval.}%
\label{fig:ANOVA_Family_Approval}%
\end{figure}%

Post hoc comparisons using the Tukey-Kramer test ~\ref{fig:ANOVA_Family_Approval} indicated that the mean score for the perspective of family approval (M = 0.50, SD = 0.69) was significantly different than the perspective of family's disapproval (M = -0.19, SD = 0.78) and different than the doubt about fammily's approval or disapproval (M = 0.11, SD = 0.70).  

\begin{figure}%
\includegraphics[width=\textwidth]{img/tukey_Family_Approval}%
\caption{Tukey test for the treatments related to the attribute Family_Approval.}%
\label{fig:TUKEY_Family_Approval}%
\end{figure}%
% \Family_Approval %

\subsection{Discussion}\label{subsec:discussion}%
Análise dos Resultados (Maristela, Roberto e Guilherme

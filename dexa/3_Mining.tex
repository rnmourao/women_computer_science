\section{Data Mining}\label{sec:mining}%
%3. Teoria sobre a análise dos Dados (Roberto e Guilherme).%

Data mining can be defined as the process of discovering insightful patterns and predictive models from data~\cite{Zaki2014}, trying to make sense of usually large amounts of data in some domain~\cite{Cios2007}. \gnramos{In this work, the main interest is the gender gap in the fields of the ``hard sciences'', so we attempt to understand the profiles of women intending to enroll in undergraduate studies, specially those interested in Computer Science.}

From 2011 to 2014, 3707 people were polled, responding a questionnaire\footnote{\url{https://github.com/rnmourao/women\_computer\_science/blob/master/data/questionnaire.pdf}}. The idea was to obtain information on their perceptions of their future undergraduate studies.

\subsection{Poll}\label{sec:mining:poll}%
The poll had 14 questions which translated into 36 variables.
\gnramos{descrição das perguntas, motivações, tipos de respostas (categoricas, numericas...)}

\subsection{Statistical Analysis}\label{sec:mining:stat}%
The usual pre-processing tasks were quickly done; all questionnaires were created and processed by CESPE\footnote{\url{http://www.cespe.unb.br/}}, a specialized research center, and the results given in simple spreadsheets. The data for all years was consolidated in a single spreadsheet\footnote{\url{https://github.com/rnmourao/women\_computer\_science/blob/master/data/raw.xlsx}}
which was then processed in the \texttt{R} programming language.

The script cleans up the data (empty columns, whitespaces, etc.) and begins to process it for analysis. The first step is to consider only the data for the 3680 female respondents. The data is distributed throughout the years as illustrated in Figure~\ref{fig:RespondentsPerYear}. \gnramos{Descrição desta informação}.

\begin{figure}%
\includegraphics[width=\textwidth]{img/RespondentsPerYear}%
\caption{Number of respondents per year.}%
\label{fig:RespondentsPerYear}%
\end{figure}%

Figure~\ref{fig:EducationalStage} shows how the respondents were distributed by educational stage, indicates that \gnramos{qual a relevância desta informação?},

\begin{figure}%
\includegraphics[width=\textwidth]{img/EducationalStage}%
\caption{Educational Stages of respondents.}%
\label{fig:EducationalStage}%
\end{figure}%

Figure~\ref{fig:FieldOfInterest} shows the respondents' interest in different fields of study. This results indicate that, throughout the years, the percentages for each choice remains more or less the same with an average of $41\%$ for \emph{Biology-Health Sciences}, $22\%$ for \emph{Exact Sciences} and $33\%$ for \emph{Human Sciences}.

\begin{figure}%
\includegraphics[width=\textwidth]{img/FieldOfInterest}%
\caption{Respondents' fields of interest.}%
\label{fig:FieldOfInterest}%
\end{figure}%

Figure~\ref{fig:InterestInCS.pdf} shows the respondents' interest in different fields of study. This results indicate that, throughout the years, the percentages for each choice remains more or less the same with an average of $41\%$ for \emph{Biology-Health Sciences}, $22\%$ for \emph{Exact Sciences} and $33\%$ for \emph{Human Sciences}.

\begin{figure}%
\includegraphics[width=\textwidth]{img/InterestInCS.pdf}%
\caption{Respondents interested in Computer Science.}%
\label{fig:InterestInCS.pdf}%
\end{figure}%

% ANOVA %%%%%%%%%%%
As described earlier, the data consisted of a group of 3,161 questionnaires with 18 questions each. There was 3,707 questionnaires, but some were excluded for many reasons: some where answered by men, others where answered by College or Adult Education Program students. None of these were adequate for the study. 

One of questionnaires' questions was related to the intention of the respondent for applying to a Computer Science Undergraduate Course. The possible answers were `No', `Maybe' and `Yes'. The answers for this question were converted to a scale, where `No' received the value -1, `Maybe' received the value 0, and `Yes' receive the value 1. The attribute received the name `CS.Choice'.

The others questions were converted to another 35 attributes, as explained earlier.

The first step was to remove all questionnaires of male respondents. After this, the attribute Gender was removed. The questionnaires answered by students of College and Adult Education Program were removed, too.

Thus, the number of questionnaires was reduced to 3,161, with 745 students that answered `No' to intention in applying for a Computer Science Undergraduate Course, 1,238 students that marked `Maybe', and 1,178 that would like to take a CS course. The grand mean, i.e., the genereal mean for the attribute CS.Choice was 0.14. This number shows a general uncertainty about the decision for application on a Computer Science Course. 

The next step was to perform a One-Way ANOVA with the treatments of all attributes, removing one by one, based on the treatments' significance~\cite{Chambers1990}. The remained attributes where Computer.Digital.Inclusion.Centre, Computer.Friends, Computer.Home, Computer.Lan.House, Computer.School, Create.Softwares, Educational.Stage, Family.Approval, Field.Interest, Has.Low.Math, Man.Majority, Need.Higher.Education, Only.Teaches.Software, Play.Games, Use.Database, and Year.

After this, a One-Way ANOVA was performed to compare the effect of each attribute on the students' intention of applying for a Computer Science Course. It was selected three attributes based on the treatments with highest means: Family.Approval, Computer.Home, and Computer.Digital.Inclusion.Centre.

Lastly, was executed a Two-Way ANOVA to evaluate the interactions between the attributes' treatments, taking the attributes two by two.   

A Tukey-Kramer post hoc test was used after each ANOVA, for the cases when there was more than two treatments.


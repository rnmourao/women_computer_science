\section{Data on Brazilian Middle and High School Girls}\label{sec:mining}%
Association rules are easy to understand, for example:``\emph{if A occurred, then B occurs}''. This can be more formally expressed as show in Equation~\ref{eq:eq.Rule}, which denotes the rule. The itemsets before the arrow are called \emph{antecedent}. Analogously, the itemsets after the arrow are called \emph{consequent}~\cite{Hastie2009}.

\begin{equation}
{A} => {B}
\label{eq:eq.Rule}%
\end{equation}

The confidence is the conditional probability $P(B|A)$, i.e., the probability of \emph{A} will occur, since \emph{B} occurred~\cite{Hastie2009}. \gnramos{Therefore, when inquiring students for their perspectives in an undergraduate course, one might find a rule such as ``\emph{if the student knows a CS professional has good wages, then he/she will consider applying for a CS degree}''}.

Another measure used in this paper, in order to select interesting rules, is \emph{lift}, which computes the ratio between the rule's confidence and the support of the itemset in the rule consequent. It represents the level of association between the antecedent and the consequent~\cite{tan2006introduction}, thus rules with greater lift are more interesting for knowledge discovery.

From 2011 to 2014, 3707 people were polled in Brazil's Federal District, responding a questionnaire whose goal was to obtain information on their perceptions of their future undergraduate studies. The poll had 18 questions, concerning gender, education, academic interests, possibilities before, during and after the pursuit of a degree in Computer Science, and how computers were used. These were translated into 36 variables and the data was investigated.

\subsection{Statistical Analysis}\label{sec:mining:stat}%
The usual pre-processing tasks were quickly done; all questionnaires were created and processed by CESPE\footnote{\url{http://www.cespe.unb.br}}, a specialized research center, and the results given in simple spreadsheets. The data for all years was consolidated in a single spreadsheet, which was then processed in the \texttt{R} programming language. The questionnaires, data and scripts used on this work are freely available online\footnote{\url{http://goo.gl/oJYrjh}}.

The script cleans up the data (empty columns, whitespaces, etc.) and begins to process it for analysis. The first step was to remove the data for the year 2011 because the poll for that year had different questions than those of the following years. Then we selected the data subset for group of interest: Middle or High School girls who have answered if they have interest in a CS major. This resulted in data for 1709 respondents of which 878 in the year 2012, 469 in 2013, and 362 in 2014.

The amount of data from 2012 is roughly twice the size of the other years, \gnramos{Qual o motivo de haver mais dados de 2012?}. Nevertheless, the answers have a similar distribution throughout the years, as can be seen in the following figures, such as Figure~\ref{fig:EducationalStage} which shows how they were distributed by educational stage, indicates that \gnramos{qual a relevância desta informação?},

\begin{figure}[h!]%
\includegraphics[width=\textwidth]{img/EducationalStage}%
\caption{Educational Stages of respondents.}%
\label{fig:EducationalStage}%
\end{figure}%

Figure~\ref{fig:FieldOfInterest} shows the respondents' interest in different fields of study. These results indicate that, throughout the years, the percentages for each choice remains more or less the same with an average of $41\%$ for \emph{Biology-Health Sciences}, $22\%$ for \emph{Exact Sciences} and $33\%$ for \emph{Human Sciences}.

\begin{figure}[h!]%
\includegraphics[width=\textwidth]{img/FieldOfInterest}%
\caption{Respondents' fields of interest.}%
\label{fig:FieldOfInterest}%
\end{figure}%

Figure~\ref{fig:WouldEnrollInCS} shows the respondents' interest in enrolling in a Computer Science major. In average, the data shows that $31\%$ of the girls \emph{have interest} while $28\%$ \emph{have no interest} and $41\%$ \emph{have doubt}.

\begin{figure}[h!]%
\includegraphics[width=\textwidth]{img/WouldEnrollInCS}%
\caption{Respondents interested in Computer Science.}%
\label{fig:WouldEnrollInCS}%
\end{figure}%
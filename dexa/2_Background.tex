\section{Background \& Related Works}\label{sec:background}%

\subsection{Girls in Computer Science}\label{subsec:background:girls}%
This section is divided into two parts. The first presents studies about women in Computer Sciences majors in Brazil and around the world. The second part presents a study carried out exclusively at the Department of Computer Science at the University of Brasilia.

\subsection{Overview of women in computing }
There various studies that address the theme of gender in the field of computers, some of which are presented as follows.
Jane et al. in \cite{jane_2016} and Sappa et al. in \cite{sappa_2013} present a study about stereotyping in Computer majors, which shows that in the United States there is a higher rate of men in the field of computers than women. Another gender study, Putnik et al.  \cite{zoran_2017}, presents data from Yugoslavia, comparing rates of males and females in the field, observing a higher rate of males than females. In \cite{keinan_2017} \cite{moshe_2015} data shows that in the United States, the rate of women graduating in Bachelor programs in the field of computers was nearly 40\% in 1984, which dwindled to 20\% by 2006. Vivian Anette in \cite{vivian_2007} affirms that Computer Science is concurrently seen as a science of technology that has excluded women.

 In \cite{moshe_2015}  the Computer Science Research Association in North America reports data from 2013 and 2014 in the United States, citing that only 14.7\% of those graduating in Computer Science fields were women. 

In Brazil, there is a similar phenomenon, as [4] presents a study on female participation in university majors in Computer Sciences in Brazil, based on the Higher Education Census data from the Ministry of Culture and Education, between the years of 2000 and 2013. Two important issues were raised. Firstly, within this period, while the number of graduating male students increased 98\%, the number of female graduates decreased 8\%. Second, the maximum percentage of graduating students in computer majors during this period, in Brazil, in different majors within the field of computers (Information Systems, Computer Science, and Computer Engineering) was a mere 20\%.  


\subsection{Women Majoring in Computers at the University of Brasilia}

The Department of Computer Sciences at the University of Brasilia (UnB) has three undergraduate majors: a Bachelor?s in Computer Science, Teaching Credential in Computers, and Computer Engineering. 
    The following graphs present data of students enrolled in the Bachelor?s in Computer Science (Fig.1) from UnB, since 1983, when the major was inaugurated. The Teaching Credential in Computers (Fig.2) and the Computer Engineering major (Fig.3) were initiated in 1997 and 2008, respectively. The data analyzed was taken from the Undergraduate Academic Information System (acronym in Portuguese - SIGRA), from UnB, which contains detailed information from undergraduate students at UnB.

Based on the statistics of students enrolled in Computer Sciences, in Fig. 1, 2 and 3, it is verified that the number of women majoring in computers at UnB (Bachelors in Computer Science, Teaching Credential in Computer Science and Computer Engineering) is, in fact, very low. An important observation is that, in the initial years of the Major in Computer Sciences (Fig.1), the quantity of male and female students enrolled was similar: in 1993 there were ten women, which shows that 33\% of those enrolled were female. In 2013 the enrollment of girls was 6 and boys 89, which tallies to only 6\% of those enrolled being female. With this data, it is possible to observe that throughout the years the disparity between the number of male and female students enrolled majors in Computer Sciences at UnB has increased, with males being the greater number.

\subsection{Data Analysis}\label{subsec:background:data}%

Data analysis includes, among other things, procedures for analyzing data and techniques for interpreting their results~\cite{Tukey1962}.

The \emph{Apriori} algorithm provides a computationally feasible solution for finding association rules within data~\cite{Hastie2009}, with an original approach to do fewer passes in the database: it generate candidate itemsets using only the itemsets found large in the previous pass~\cite{Agrawal1994}. The goal is to find the rules with the highest \emph{confidence}, despite their number of occurrences (\emph{support}) \cite{taniar_exception_2008}. 

The equation \ref{eq:eq.Rule} denotes a rule, and it is read as: "if A occurred, then B occurs." The itemsets before the arrow are called \emph{antecedent}. Analogously, the itemsets after the arrow are called \emph{consequent}~\cite{Hastie2009}.

\begin{equation}
{A} => {B}
\label{eq:eq.Rule}%
\end{equation}

The confidence is the conditional probability P(B|A), i.e., the probability of \emph{A} will occur, since \emph{B} occurred~\cite{Hastie2009}.

Another measure used in this paper, to select interesting rules, was \emph{lift}, which computes the ratio between the rule's confidence and the support of the itemset in the rule consequent. It represents the level of association between the antecedent and the consequent.~\cite{tan2006introduction}

\gnramos{Inserir detalhes do algoritmo?}

\gnramos{Breve relação das técnicas utilizadas}%

\subsection{Related Works}\label{sec:background:related}%

In~\cite{papastergiou_are_2008}, 358 Greek high school students' intentions and motivation for pursuing academic studies in Computer Science were investigated. This research looked into the influence of family and academic environment on their career choices, their perception of a professional career in CS, the attendance in school courses related to the field, the use of a computer at home, and their self-efficacy beliefs regarding computers. The analysis showed that lack of an early use of a computer at home and in school seems to be the main factor discouraging students decide to study CS, and this is very clear considering only the girls' data.

The main reason why Greek students choose to study CS is employability; and the girls' motivation is more closely related to extrinsic factors, such as improved employment prospects than to intrinsic ones (like a personal interest in the field). The majority of students were not explicitly encouraged to or deterred from studying CS by others (including family and friends); they do not consider CS or IT related professions as a mainly masculine option.

In~\cite{anderson_because_2008}, a study of possible factors related to low rates of female participation in education pathways leading to information and communications technology (ICT) professions was done considering data from a three-year period. The survey using binary options, such as ``I am very interested in computers'' and ``I am not interested in computers'', was presented to 1,453 high school girls in their Senior year. The study identified two factors associated with a woman's aversion to an ICT profession: the perception that the subject is boring, and an intense dislike of computers.

\gnramos{Qual o mérito deste trabalho?}

Our work searches for insights on students' motivation for academic studies in Computer Science, with a clear gender bias (looking and girls only) and analyzing a larger data set of Brazilian students.

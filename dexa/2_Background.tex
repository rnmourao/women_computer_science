\section{Related Works \& Data Analysis}\label{sec:background}%

There have been several studies concerning women in Computer Sciences majors in different contexts, from those looking at a global scale to others focusing on a single country or even smaller venues. These were conducted with many approaches; we are interested in using Data Mining to try to understand the reasons for few women enrolling the University of Brasília's CS related courses.

\subsection{Overview of women in computing}%
There various studies that address the theme of gender in the field of computers, some of which are presented as follows.
Jane et al. in~\cite{jane_2016} and Sappa et al. in~\cite{sappa_2013} present a study about stereotyping in Computer majors, which shows that in the United States there is a higher rate of men in the field of computers than women. Another gender study, Putnik et al.~\cite{zoran_2017}, presents data from Yugoslavia, comparing rates of males and females in the field, observing a higher rate of males than females. In~\cite{keinan_2017,moshe_2015} data shows that in the United States, the rate of women graduating in Bachelor programs in the field of computers was nearly 40\% in 1984, which dwindled to 20\% by 2006. Vivian Anette in~\cite{vivian_2007} affirms that Computer Science is concurrently seen as a science of technology that has excluded women. In~\cite{moshe_2015}  the Computer Science Research Association in North America reports data from 2013 and 2014 in the United States, citing that only 14.7\% of those graduating in Computer Science fields were women.

In Brazil, there is a similar phenomenon, as \gnramos{O texto original diz [4], acho que é ~\cite{maia_2016}} presents a study on female participation in university majors in Computer Sciences in Brazil, based on the Higher Education Census data from the Ministry of Culture and Education between the years of 2000 and 2013~\cite{inep2014}. Two important issues were raised. Firstly, within this period, while the number of graduating male students increased 98\%, the number of female graduates decreased 8\%; and the maximum percentage of graduating students in computer majors during this period, in Brazil, in different majors within the field of computers (Information Systems, Computer Science, and Computer Engineering) was a mere 20\%.

\subsection{Women Majoring in Computers at the University of Brasília}

The Department of Computer Sciences\footnote{\url{http://cic.unb.br}} at the University of Brasília (UnB) has three undergraduate majors: a Computer Science, \gnramos{Licentiate in Computing}, and Computer Engineering. The following figures present data of students enrolled in Computer Science \gnramos{esta figura não foi inclusa (Fig.1)} from UnB, since 1983, when the degree became available. The \gnramos{Licentiate in Computing} \gnramos{esta figura não foi inclusa (Fig.2)} and the Computer Engineering majors \gnramos{esta figura não foi inclusa (Fig.3)} began in 1997 and 2008, respectively. The data analyzed was taken from the university's Undergraduate Academic Information System, which contains detailed information from undergraduate students at UnB.

The numbers for the UnB's students show that there are very few women enrolled in these courses. In the initial years of the major in Computer Sciences (see \gnramos{esta figura não foi inclusa (Fig.1)}), the difference between quantities of male and female students enrolled was no so great, 33\% of those students were female. In contrast, of the 95 students enrolled in 2013, only 6\% were female. The disparity has increased over the years, and we wish to investigate this to address the issue.

\subsection{Data Analysis}\label{sec:background:related}%

Data analysis includes, among other things, procedures for analyzing data and techniques for interpreting their results~\cite{Tukey1962}, while Data Mining is the process of discovering insightful patterns and predictive models from data~\cite{Zaki2014}, trying to make sense of usually large amounts of information in some domain~\cite{Cios2007}. \gnramos{In this work, the primary interest is the gender gap in the fields of the ``hard sciences'', so we attempt to understand the profiles of women intending to enroll in undergraduate studies, especially those interested in Computer Science.}

Papastergiou used descriptive statistics, principal component analysis and analysis of variance (ANOVA) to investigate 358 Greek high school students' intentions and motivation for pursuing academic studies in Computer Science were investigated~\cite{papastergiou_are_2008}. This looked into the influence of family and academic environment on their career choices, their perception of a professional career in CS, the attendance in school courses related to the field, the use of a computer at home, and their self-efficacy beliefs regarding computers. The analysis showed that lack of an early use of a computer at home and in school seems to be the main factor discouraging students decide to study CS, and this is very clear considering only the girls' data.

The main reason why Greek students choose to study CS is employability; and the girls' motivation is more closely related to extrinsic factors, such as improved employment prospects than to intrinsic ones (like a personal interest in the field). The majority of students were not explicitly encouraged to or deterred from studying CS by others (including family and friends); they do not consider CS or IT related professions as a mainly masculine option.

Anderson et al. applied means and non-parametric statistics and Mann–Whitney $U$ test comparison to study of possible factors related to low rates of female participation in education pathways leading to information and communications technology (ICT) professions, considering data from a three-year period~\cite{anderson_because_2008}. The survey using binary options, such as ``I am very interested in computers'' and ``I am not interested in computers'', was presented to 1,453 high school girls in their Senior year. The study identified two factors associated with a woman's aversion to an ICT profession: the perception that the subject is boring, and an intense dislike of computers.

\gnramos{These approaches provide insightful results, but do not scale well nor provide a straightforward interpretation of the results}. Association rules mining, on the other hand, finds interesting relationships in data, with good scalability and easily understandable results~\cite{Cios2007}. The \emph{Apriori} algorithm provides a computationally feasible solution for finding association rules within data~\cite{Hastie2009}, with an original approach to do fewer passes in the database: it generates candidate itemsets using only the itemsets found large in the previous pass~\cite{Agrawal1994}. The goal is to find the rules with the highest \emph{confidence} (how often the rule has been found to be true), despite their number of occurrences (\emph{support})~\cite{taniar_exception_2008}.

Our work searches for insights on students' motivation for academic studies in Computer Science, with a clear gender bias (looking and girls only) and analyzing a larger data set of Brazilian students.

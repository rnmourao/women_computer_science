\section{Background \& Related Works}\label{sec:background}%

\subsection{Girls in Computer Science}\label{subsec:background:girls}%
2. Meninas e a Computação (Maristela)

Middle School: Ensino Fundamental II
High School: Ensino Médio

\subsection{Data Analysis}\label{subsec:background:data}%

Data analysis includes, among other things, procedures for analyzing data and techniques for interpreting their results~\cite{Tukey1962}.

\gnramos{Breve relação das técnicas utilizadas}%
ANOVA ~\cite{Hastie2009}

A useful statistical technique is Analysis of Variance, also known as ANOVA. The ANOVA tests if two or more treatments have the same mean. This assumption is called \emph{null hipothesis}. The \emph{alternate hipothesis} declares that the treatments have different means \cite{Vieira2006}.

Despite its significance power, i.e., the ability to indicate that are differences between the treatments, ANOVA doesn't say which treatments are different each other. In the case of only two treatments, the answer is trivial, But it is necessary to execute one more test when there are three or more treatments. The \emph{Tukey HSD} test does \emph{multiple comparisons} between the treatments and determine which treatments have the same means and which don't \cite{moore2009practice}.

Frequently, data may be usefully and accurately treated by the analysis of variance (through separation of the variance attributes to a group of causes from that attributed to other groups)~\cite{Fisher1934}. This \emph{analysis of variance} (ANOVA)can be achieved through several approaches, for example through a hypothesis test comparing the two models, considering the null hypothesis that both models fit the data equally well in contrast to  the full model is superior~\cite{James2013}.

When the treatments have different sizes of replications, the ANOVA is quite more complex\cite{ott2008introduction}, because this relative weight of the treatments may disturb their significance levels \cite{schuessler1971analyzing}. Yates \cite{herr_history_1986} create an alternative formula called Standard Parametric (STP), which removes the effects of unequal sample sizes from the treatments.

\gnramos{Trabalhos correlacionados}%

Papastergiou~\cite{papastergiou_are_2008} investigated 358 male and female greek high school students' intentions and motivation towards and against pursuing academic studies in Computer Science (CS). The research pertained the influence of the family and the scholastic environment on students' career choices, the students' perceptions of CS profession as well as students' attendance at CS courses at school, the computer use in the home and the self-efficacy beliefs regarding computers. The study revealed that lack of opportunities for early use of computer in home and school seems to be the main factor of discouragement to boys and girls decide to study CS, having a much greater impact on girls; employability seems to be the main reason why Greek students decide to study CS; girls' motivation towards studying CS is related to extrinsic factors, e.g., enhanced employment prospects, rather than intrinsic ones, e.g., personal interest in CS; the majority of the students were not found to be explicitly encouraged to or deterred from studying CS by their family, friends, CS teachers or other adults; the students of the sample do not consider CS and the IT profession as masculine fields; within the home environment, girls were found to be less likely than boys to use a computer.

Anderson et al.~\cite{anderson_because_2008}, during three years, studied possible factors related to  low participation rates by females in education pathways leading to professional level information and communications technology (ICT) professions. The survey was offered to 1,453 senior high school girls, and used binary options, e.g., "I am very interested in computers" and "I am not interested in computers". The study identified that are two factors associated with women's aversion toward a ICT profession: the
perception that these subjects are boring, and a strongly aversion to computers. 
\section{Conclusion}\label{sec:conclusion}%

In recent years, the area of Computer Sciences has had the participation of few women in the profession, showing that girls have not shown interest in majoring or becoming professionals in this field. Given this scenario, research aimed at understanding their motives for not choosing a major or career in computers is important to promoting actions that will reduce the disparity between boys and girls entering the field, by increasing participation on the part of the girls. 

Although the research was carried out in the Capitol of Brazil, the Federal District, it is important to note that the methodology and scripts of the R in this article can be replicated in other contexts. As can be observed, this problem of gender disparity is not unique to Brazil, but occurs in other countries as well. 

Future actions include: the application of this research to boys, to compare the results to the girls; the application of other techniques of knowledge discovery; the application of questionnaire to students who enroll in majors in the field of computers and compare their responses to those that we have by high school students interested in Computer Sciences.  

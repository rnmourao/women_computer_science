\section{Introduction}\label{sec:intro}%
\gnramos{Falar sobre a questão das mulheres na Computação (Maristela)}%

\gnramos{Falar da análise realizada (Roberto e Guilherme)}%
In this paper, we investigate the relation between intention to apply for a Computer Science undergraduate course and affinity with computers and Computer Science of Middle and High School girls. The data is originated from a 18 questions' poll executed between the years of 2012 and 2014, to teenage girls (N = 3,707) in Distrito Federal, Brazil, and the questions were converted to 36 variables. One of these variables was the girl's intention to applying for a Computer Science course, having an escale from -1 to 1, with -1 indicating the girl's rejection, 0, indicating the girl's doubt, and 1, indicating the girl's intention. Each of remaining variables had its possible values converted to treatments. An ANOVA additive model was executed to remove non-significant attributes. After this, a One-Way ANOVA was performed for the treatments of each attribute, to check main effects. A Tukey HSD test was used for multiple comparisons. Finally, a Two-Way ANOVA and a Tukey HSD were performed between the attributes' treatments. The attributes related to girl's family opinion about her intention to applying for a CS course and the girl's habit of using a LAN Gaming House were selected to analysis because the interaction between these two showed the highest mean for the response variable.

\gnramos{Falar dos dados (Maristela)}%

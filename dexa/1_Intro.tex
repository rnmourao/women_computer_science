\section{Introduction}\label{sec:intro}%

In Brazil, the choice of an undergraduate major in the area of Computer Sciences is not among the top choices for girls in high school when contemplating a career. As \cite{maia_2016} presents, between 2000 and 2013 in Brazil, an average of only 17\% of all graduates in various Computer Science majors were women. This research covered majors in Computer Science, Computer Engineering, and Information Systems, among others. Particularly, in the Federal District, at the University of Brasilia, which currently has approximately 30,000 students enrolled in undergraduate programs, the reality is even worse, where in the past ten years, according to \cite{couto_2014} only 10\% were women. 

    Responding to the low incidence of women in Computer Science majors, recently researchers have given much thought about how to improve this scenario and proposed strategies to encourage girls to pursue a profession in the Computer Sciences   \cite{cohoon_2002} \cite{couto_2014}  \cite{gurer_2002}  \cite{maia_2016}. Brazil and other countries have developed initiatives to debate this issue. Specifically, the Institute of Electrical and Electronics Engineers (IEEE) has a program which addresses the problem: the IEEE Women in Engineering (WIE) \cite{wie2017}. The WIE is a major professional and international organization dedicated to promoting women scientists and engineers. Another prominent program in helping women in the area of Computers is ?Girls who Code? \cite{girlsWC_2017}, which has over 40,000 members and various initiatives to increase the participation of girls in Computer Sciences over different regions in the United States. Another initiative from the United States is the ?Grace Hopper Celebration of Women in Computer Sciences? event, which is the biggest event worldwide for discussing the theme of women in the field. In 2016 alone 15,000 people from 87 countries participated in the 700 presentations \cite{GHC_2017}.
    
    In Brazil, since 2007, the Brazilian Society of Computing Conference held the Women in Information and Technology Workshop (WIT), to discuss the theme. Brazilian governmental agencies, such as the Ministry of Science and Technology released calls for submissions of research projects specifically related to the education of girls in Computing or Physical Science majors \cite{cnpq_2017}. Aiming to gather information about the perceptions of high school girls regarding computer science, the Department of Computer Science at the University of Brasilia, developed the project, Meninas.comp: computação  tambem é coisa de menina, Girls.comp: computer science is a girl thing too.  
    

\gnramos{Falar da análise realizada (Roberto e Guilherme)}%

Between the years of 2012 and 2014, we contacted thousands of girls in Middle and High School to investigate the relationship between their intention to apply for an undergraduate course in Computer Science and their affinity with the field and computing tools. We used the Apriori algorithm on the collected data, searching for interesting association rules and insights on the girls' interest in CS and their background.

This following sections of this work are organized as follows: Section~\ref{sec:background} presents related work and background information on this approach, Section~\ref{sec:mining} describes details of the data mining applied, Section~\ref{sec:results} provides our experimental results and our finds and Section~\ref{sec:conclusion} presents concluding remarks.

\section{Introduction}\label{sec:intro}%
\gnramos{Falar sobre a questão das mulheres na Computação (Maristela)}%

\gnramos{Falar da análise realizada (Roberto e Guilherme)}%
In this paper, we investigate the relation between intention to apply for a Computer Science undergraduate course and affinity with computers and Computer Science of Middle and High School girls. The data is originated from a 18 questions' poll executed between the years of 2012 and 2014, to teenage girls (N = 3,709) in Distrito Federal, Brazil. 
The questions were converted to 36 variables, where one of these variables, called as \emph{CS.Interest},  was the girl's intention to applying for a Computer Science course, with three possible answers: \emph{Yes}, \emph{Maybe}, or \emph{No}. An association rule mining using the Apriori's algorithm was performed to find possible relations between the \emph{CS.Interest} variable the other ones. 
As the percentage of respondents interested in applying for a Computer Science Program was 31\%, approximately, a higher confidence metric was considered more important than a higher support \cite{taniar_exception_2008}. Furthermore, the other answers of \emph{CS.Interest} variable (Maybe and No) were keeped, in order to check the existence of opposite, but, interesting rules.

\gnramos{Falar dos dados (Maristela)}%
